
\chapter{敬濟元夜戲嬌姿 惠祥怒詈來旺婦}

詩曰:
\begin{quote}
銀燭高燒酒乍醺,當筵且喜笑聲頻。
蠻腰細舞章台柳,素口輕歌上苑春。
香氣拂衣來有意,翠花落地拾無聲。
不因一點風流趣,安得韓生醉後醒。
\end{quote}

話說一日,天上元宵,人間燈夕,西門慶在廳上張掛花燈,鋪陳綺席。正月十六,合家歡樂飲酒。西門慶與吳月娘居上,其餘李嬌兒、孟玉樓、潘金蓮、李瓶兒、孫雪娥、西門大姐都在兩邊同坐,都穿著錦繡衣裳。春梅、玉簫、迎春、蘭香一般兒四個家樂,在旁𢺈箏歌板,彈唱燈詞。獨於東首設一席與女婿陳敬濟坐。果然食烹異品,果獻時新。小玉、元宵、小鸞、繡春都在上面斟酒。那來旺兒媳婦宋蕙蓮卻坐在穿廊下一張椅兒上,口裡嗑瓜子兒。等的上邊呼喚要酒,他便揚聲叫:「來安兒,畫童兒,上邊要熱酒,快趲酒上來!賊囚根子,一個也沒在這裡伺候,都不知往那去了!」只見畫童燙酒上去。西門慶就罵道:「賊奴才,一個也不在這裡伺候,往那去來?賊少打的奴才!」小廝走來說道:「嫂子,誰往那去來?就對著爹說,吆喝教爹罵我。」蕙蓮道:「上頭要酒,誰教你不伺候?關我甚事!不罵你罵誰?」畫童兒道:「這地上乾乾凈凈的,嫂子嗑下恁一地瓜子皮,爹看見又罵了。」蕙蓮道:「賊囚根子!六月債兒熱,還得快就是。甚麼打緊,便當你不掃,丟著,另教個小廝掃。等他問我,只說得一聲。」畫童兒道:「耶嚛,嫂子,將就些罷了,如何和我合氣!」於是取了笤帚來,替他掃瓜子皮兒,不題。

卻說西門慶席上,見女婿陳敬濟沒酒,吩咐潘金蓮去遞一巡兒。這金蓮連忙下來,滿斟杯酒,笑嘻嘻遞與敬濟,說道:「姐夫,你爹吩咐,好歹飲奴這杯酒兒。」敬濟一壁接酒,一面把眼兒斜溜婦人,說:「五娘請尊便,等兒子慢慢吃!」婦人將身子把燈影著,左手執酒,剛待的敬濟將手來接,右手向他手背只一捻,這敬濟一面把眼瞧著眾人,一面在下戲把金蓮小腳兒踢了一下。婦人微笑,低聲道:「怪油嘴,你丈人瞧著待怎麼?」兩個在暗地裡調情頑耍,眾人倒不曾看出來。不料宋蕙蓮這婆娘,在槅子外窗眼裡,被他瞧了個不耐煩。口中不言,心下自忖:「尋常在俺們跟前,到且是精細撇清,誰想暗地卻和這小夥子兒勾搭。今日被我看出破綻,到明日再搜求我,自有話說。」正是:
\begin{quote}
誰家院內白薔薇,暗暗偷攀三兩枝。
羅袖隱藏人不見,馨香惟有蝶先知。
\end{quote}

飲酒多時,西門慶忽被應伯爵差人請去賞燈。吩咐月娘:「你們自在耍耍,我往應二哥家吃酒去來。」玳安、平安兩個跟隨去了。

月娘與眾姊妹吃了一回,但見銀河清淺,珠鬥爛斑,一輪團圓皎月從東而出,照得院宇猶如白晝。婦人或有房中換衣者,或有月下整妝者,或有燈前戴花者。惟有玉樓、金蓮、李瓶兒三個並蕙蓮,在廳前看敬濟放花兒。李嬌兒、孫雪娥、西門大姐都隨月娘後邊去了。金蓮便向二人說道:「他爹今日不在家,咱對大姐姐說,往街上走走去。」蕙蓮在旁說道:「娘們去,也攜帶我走走。」金蓮道:「你既要去,你就往後邊問聲你大娘和你二娘,看他去不去,俺們在這裡等著你。」那蕙蓮連忙往後邊去了。玉樓道:「他不濟事,等我親自問他聲去。」李瓶兒道:「我也往屋裡穿件衣裳,只怕夜深了冷。」金蓮道:「李大姐,你有披襖子,帶件來我穿,省得我往屋裡去。」那李瓶兒應諾去了。獨剩下金蓮一個,看著敬濟放花兒。見無人,走向敬濟身上捏了一把,笑道:「姐夫原來只穿恁單薄衣裳,不害冷麼?」只見家人兒子小鐵棍兒笑嘻嘻在跟前,舞旋旋的且拉著敬濟,要炮丈放。這敬濟恐怕打攪了事,巴不得與了他兩個元宵炮丈,支他外邊耍去了。於是和金蓮嘲戲說道:「你老人家見我身上單薄,肯賞我一件衣裳兒穿穿也怎的?」金蓮道:「賊短命,得其慣便了,頭裡頭躡我的腳兒,我不言語,如今大膽,又來問我要衣服穿!我又不是你影射的,何故把與你衣服穿?」敬濟道:「你老人家不與就罷了,如何扎筏子來唬我?」婦人道:「賊短命,你是城樓上雀兒,好耐驚耐怕的蟲蟻兒!」正說著,見玉樓和蕙蓮出來,向金蓮說道:「大娘因身上不方便,大姐不自在,故不去了。教娘們走走,早些來家。李嬌兒害腿疼,也不走。孫雪娥見大姐姐不走,恐怕他爹來家嗔他,也不出門。」金蓮道:「都不去罷,只咱和李大姐三個去罷。等他爹來家,隨他罵去!再不,把春梅小肉兒和上房裡玉簫,你房裡蘭香,李大姐房裡迎春,都帶了去。」小玉走來道:「俺奶奶已是不去,我也跟娘們走走。」玉樓道:「對你奶奶說了去,我前頭等著你。」良久,小玉問了月娘,笑嘻嘻出來。

當下三個婦人,帶領著一簇男女。來安、畫童兩個小廝,打著一對紗吊燈跟隨。女婿陳敬濟踹著馬台,放煙火花炮,與眾婦人瞧。宋蕙蓮道:「姑夫,你好歹略等等兒。娘們攜帶我走走,我到屋裡搭搭頭就來。」敬濟道:「俺們如今就行。」蕙蓮道:「你不等,我就惱你一生!」於是走到屋裡,換了一套綠閃紅緞子對衿衫兒、白挑線裙子。又用一方紅銷金汗巾子搭著頭,額角上貼著飛金並面花兒,金燈籠墜耳,出來跟著眾人走百媚兒。月色之下,恍若仙娥,都是白綾襖兒,遍地金比甲。頭上珠翠堆滿,粉面朱唇。敬濟與來興兒,左右一邊一個,隨路放慢吐蓮、金絲菊、一丈蘭、賽月明。出的大街市上,但見香塵不斷,遊人如蟻,花炮轟雷,燈光雜彩,簫鼓聲喧,十分熱鬧。遊人見一對紗燈引道,一簇男女過來,皆披紅垂綠,以為出於公侯之家,莫敢仰視,都躲路而行。那宋蕙蓮一回叫:「姑夫,你放個桶子花我瞧。」一回又道:「姑夫,你放個元宵炮丈我聽。」一回又落了花翠,拾花翠;一回又掉了鞋,扶著人且兜鞋;左來右去,只和敬濟嘲戲。玉樓看不上,說了兩句:「如何只見你掉了鞋?」玉簫道:「他怕地下泥,套著五娘鞋穿著哩!」玉樓道:「你叫他過來我瞧,真個穿著五娘的鞋兒?」金蓮道:「他昨日問我討了一雙鞋,誰知成精的狗肉,套著穿!」蕙蓮摳起裙子來,與玉樓看。看見他穿著兩雙紅鞋在腳上,用紗綠線帶兒扎著褲腿,一聲兒也不言語。

須臾,走過大街,到燈市裡。金蓮向玉樓道:「咱如今往獅子街李大姐房子里走走去。」於是吩咐畫童、來安兒打燈先行,迤邐往獅子街來。小廝先去打門,老馮已是歇下,房中有兩個人家賣的丫頭,在炕上睡。慌的老馮連忙開了門,讓眾婦女進來,旋戳開爐子頓茶,挈著壺往街上取酒。孟玉樓道:「老馮你且住,不要去打酒,俺們在家酒飯吃得飽飽來,你有茶,倒兩甌子來吃罷。」金蓮道:「你既留人吃酒,先訂下菜兒才好。」李瓶兒道:「媽媽子,一瓶兩瓶取來了,打水不渾的,夠誰吃?要取一兩壇兒來。」玉樓道:「他哄你,不消取,只看茶來罷。」那婆子方纔不動身。李瓶兒道:「媽媽子,怎的不往那邊去走走,端的在家做些甚麼?」婆子道:「奶奶,你看丟下這兩個業障在屋裡,誰看他?」玉樓便問道:「兩個丫頭是誰家賣的?」婆子道:「一個是北邊人家房裡使女,十三歲,只要五兩銀子;一個是汪序班家出來的家人媳婦,家人走了,主子把鬏髻打了,領出來賣,要十兩銀子。」玉樓道:「媽媽,我說與你,有一個人要,你賺他些銀子使。」婆子道:「三娘,果然是誰要?告我說。」玉樓道:「如今你二娘房裡,只元宵兒一個,不夠使,還尋大些的丫頭使喚。你倒把這大的賣與他罷。」因問:「這個丫頭十幾歲?」婆子道:「他今年十七歲了。」說著,拿茶來,眾人吃了茶。那春梅、玉簫並蕙蓮都前邊瞧了一遍,又到臨街樓上推開窗看了一遍。陳敬濟催逼說:「夜深了,看了快些家去罷。」金蓮道:「怪短命,催的人手腳兒不停住,慌的是些甚麼!」乃叫下春梅眾人來,方纔起身。馮媽媽送出門,李瓶兒因問:「平安往那去了?」婆子道:「今日這咱還沒來,叫老身半夜三更開門閉戶等著他。」來安兒道:「今日平安兒跟了爹往應二爹家去了。」李瓶兒吩咐媽媽子:「早些關了門,睡了罷!他多也是不來,省的誤了你的困頭。明日早來宅里,送丫頭與二娘來。你是石佛寺長老,請著你就張致了。」說畢,看著他關了大門,這一簇男女方纔回家。

走到家門首,只聽見住房子的韓回子老婆韓嫂兒聲喚。因他男子漢答應馬房內臣,他在家跟著人走百病兒去了,醉回來家,說有人挖開他房門,偷了狗,又不見了些東西,坐在當街上撒酒瘋罵人。眾婦人方纔立住了腳。金蓮使來安兒把韓嫂兒叫到當面,問道:「你為甚麼來?」韓嫂兒叉手向前,拜了兩拜,說道:「三位娘子在上,聽小媳婦告訴。」於是從頭說了一遍。玉樓眾人聽了,每人掏袖中些錢果子與他,叫來安兒:「你叫你陳姐夫送他進屋裡。」那敬濟且顧和蕙蓮兩個嘲戲,不肯搊他去。金蓮使來安兒扶到他家中,吩咐教他明日早來宅內漿洗衣裳:「我對你爹說,替你出氣。」那韓嫂兒千恩萬謝回家去了。

玉樓等剛走過門首來,只見賁四娘子,在大門首笑嘻嘻向前道了萬福,說道:「三位娘那裡走了走?請不棄到寒家獻茶。」玉樓道:「方纔因韓嫂兒哭,俺站住問了他聲。承嫂子厚意,天晚了,不到罷。」賁四娘子道:「耶嚛,三位娘上門怪人家,就笑話俺小家人家茶也奉不出一杯兒來?」生死拉到屋裡。原來上邊供養觀音八難並關聖賢,當門掛著雪花燈兒一盞。掀開門簾,擺設春台,與三人坐。連忙教他十四歲女兒長姐過來,與三位娘磕頭遞茶。玉樓、金蓮每人與了他兩枝花兒。李瓶兒袖中取了一方汗巾,又是一錢銀子,與他買瓜子兒嗑。喜歡的賁四娘子拜謝了又拜。款留不住,玉樓等起身。到大門首,小廝來興在門首迎接。金蓮就問:「你爹來家不曾?」來興道:「爹未回家哩。」三個婦人,還看著陳敬濟在門首放了兩個一丈菊和一筒大煙蘭、一個金盞銀台兒,才進後邊去了。西門慶直至四更來家。正是:
\begin{quote}
醉後不知天色暝,任他明月下西樓。
\end{quote}

卻說那陳敬濟因走百病,與金蓮等眾婦人嘲戲了一路兒,又和蕙蓮兩個言來語去,都有意了。次日早晨梳洗畢,也不到鋪子內,逕往後邊吳月娘房裡來。只見李嬌兒、金蓮陪著吳大妗子,放炕桌兒,才擺茶吃。月娘便往佛堂中燒香去了。這小伙兒向前作了揖,坐下。金蓮便說道:「陳姐夫,你好人兒!昨日教你送送韓嫂兒,你就不動,只當還教小廝送去了。且和媳婦子打牙犯嘴,不知甚麼張致!等你大娘燒了香來,看我對他說不說!」敬濟道:「你老人家還說哩,昨日險些兒子腰梁㿚瘍了哩!跟你老人家走了一路兒,又到獅子街房裡回來,該多少里地?人辛苦走了,還教我送韓回子老婆!教小廝送送也罷了。睡了多大回就天曉了,今早還扒不起來。」正說著,吳月娘燒了香來,敬濟作了揖。月娘便問:「昨日韓嫂兒為甚麼撒酒瘋罵人?」敬濟把因走百病,被人挖開門,不見了狗,坐在當街哭喊罵人,「今早他漢子來家,一頓好打的,這咱還沒起來哩。」金蓮道:「不是俺們回來,勸的他進去了,一時你爹來家撞見,甚麼樣子!」說畢,玉樓、李瓶兒、大姐都到月娘屋裡吃茶,敬濟也陪著吃了茶。後次大姐回房,罵敬濟:「不知死的囚根子!平白和來旺媳婦子打牙犯嘴,倘忽一時傳的爹知道了,淫婦便沒事,你死也沒處死!」

卻說那日,西門慶在李瓶兒房裡宿歇,起來的遲。只見荊千戶——新升一處兵馬都監——來拜。西門慶才起來梳頭,包網巾,整衣出來,陪荊都監在廳上說話。一面使平安兒進後邊要茶。宋蕙蓮正和玉簫、小玉在後邊院子里撾子兒,賭打瓜子,頑成一塊。那小玉把玉簫騎在底下,笑罵道:「賊淫婦,輸了瓜子,不教我打!」因叫蕙蓮:「嫂子你過來,扯著淫婦一隻腿,等我㒲這淫婦一下子。」正頑著,只見平安走來,叫:「玉簫姐,前邊荊老爹來,使我進來要茶哩。」那玉簫也不理他,且和小玉廝打頑耍。那平安兒只顧催逼說:「人坐下這一日了。」宋蕙蓮道:「怪囚根子,爹要茶,問廚房裡上竈的要去,如何只在俺這裡纏?俺這後邊只是預備爹娘房裡用的茶,不管你外邊的帳。」那平安兒走到廚房下。那日該來保妻蕙祥,蕙祥道:「怪囚,我這裡使著手做飯,你問後邊要兩鐘茶出去就是了,巴巴來問我要茶!」平安道:「我到後頭來,後邊不打發茶。蕙蓮嫂子說,該是上竈的首尾。」蕙祥便罵道:「賊淫婦,他認定了他是爹娘房裡人,俺天生是上竈的來?我這裡又做大傢伙里飯,又替大妗子炒素菜,幾隻手?論起就倒倒茶兒去也罷了,巴巴坐名兒來尋上竈的,上竈的是你叫的?誤了茶也罷,我偏不打發上去。」平安兒道:「荊老爹來了這一日,嫂子快些打發茶,我拿上去罷。遲了又惹爹罵!」

當下這裡推那裡,那裡推這裡,就耽誤了半日。比及又等玉簫取茶果、茶匙兒出來,平安兒拿茶出去,那荊都監坐的久了,再三要起身,被西門慶留住。嫌茶冷不好吃,喝罵平安另換茶上去吃了,荊都監才起身去了。西門慶進來,問:「今日茶是誰頓的?」平安道:「是竈上頓的茶。」西門慶回到上房,告訴月娘:「今日頓這樣茶出去,你往廚下查那個奴才老婆上竈?採出來問他,打與他幾下。」小玉道:「今日該蕙祥上竈。」慌的月娘說道:「這歪剌骨待死!越發頓恁樣茶上去了。」一面使小玉叫將蕙祥當院子跪著,問他要打多少。蕙祥答道:「因做飯,炒大妗子素菜,使著手,茶略冷了些。」被月娘數罵了一回,饒了他起來。吩咐:「今後但凡你爹前邊人來,教玉簫和蕙蓮後邊頓茶,竈上只管大家茶飯。」

這蕙祥在廚下忍氣不過,剛等的西門慶出去了,氣狠狠走來後邊,尋著蕙蓮,指著大罵:「賊淫婦,趁了你的心了!罷了,你天生的就是有時運的爹娘房裡人,俺們是上竈的老婆來?巴巴使小廝坐名問上竈要茶,上竈的是你叫的?你識我見的,促織不吃癩蛤蟆肉——都是一鍬土上人。你恆數不是爹的小老婆就罷了。就是爹的小老婆,我也不怕你!」蕙蓮道:「你好沒要緊,你頓的茶不好,爹嫌你,管我甚事?你如何拿人撒氣?」蕙祥聽了,越發惱了,罵道:「賊淫婦!你剛纔調唆打我幾棍兒好來,怎的不教打我?你在蔡家養的漢數不了,來這裡還弄鬼哩!」蕙蓮道:「我養漢,你看見來?沒的扯臊淡哩!嫂子,你也不是甚麼清凈姑姑兒!」蕙祥道:「我怎不是清凈姑姑兒?蹺起腳兒來,比你這淫婦好些兒。你漢子有一拿小米數兒!你在外邊,那個不吃你嘲過?你背地乾的那營生兒,只說人不知道。你把娘們還放不到心上,何況以下的人!」蕙蓮道:「我背地裡說甚麼來?怎的放不到心上?隨你壓我,我不怕你!」蕙祥道:「有人與你做主兒,你可知不怕哩!」兩個正拌嘴,被小玉請的月娘來,把兩個都喝開了:「賊臭肉們,不乾那營生去,都拌的是些甚麼?教你主子聽見又是一場兒。頭裡不曾打的成,等住回卻打的成了!」蕙祥道:「若打我一下兒,我不把淫婦口裡腸勾了也不算!我拚著這命,擯兌了你也不差廝甚麼。咱大家都離了這門罷!」說著往前去了。後次這宋蕙蓮越發猖狂起來,仗西門慶背地和他勾搭,把家中大小都看不到眼裡,逐日與玉樓、金蓮、李瓶兒、西門大姐、春梅在一處頑耍。

那日馮媽媽送了丫頭來,約十三歲,先到李瓶兒房裡看了,送到李嬌兒房裡。李嬌兒用五兩銀子買下,房中伏侍,不在話下。正是:
\begin{quote}
外作禽荒內色荒,連沾些子又何妨。
早晨跨得雕鞍去,日暮歸來紅粉香。
\end{quote}
