
\chapter{吳月娘春晝鞦韆 來旺兒醉中謗仙}

詞曰:
\begin{quote}
蹴罷鞦韆,起來整頓纖纖手。露濃花瘦,薄汗輕衣透。
見客入來,襪剗金釵溜。和羞走,倚門迴首,卻把青梅嗅。
\end{quote}

話說燈節已過,又早清明將至。西門慶有應伯爵早來邀請,說孫寡嘴作東,邀了郊外耍子去了。

先是吳月娘花園中,扎了一架鞦韆。這日見西門慶不在家,閑中率眾姊妹游戲,以消春困。先是月娘與孟玉樓打了一回,下來教李嬌兒和潘金蓮打。李嬌兒辭說身體沉重,打不的,卻教李瓶兒和金蓮打。打了一回,玉樓便叫:「六姐過來,我和你兩個打個立鞦韆。」吩咐:「休要笑。」當下兩個玉手輓定彩繩,將身立於畫板之上。月娘卻教蕙蓮、春梅兩個相送。正是:
\begin{quote}
紅粉面對紅粉面,玉酥肩並玉酥肩。
兩雙玉腕輓復輓,四隻金蓮顛倒顛。
\end{quote}

那金蓮在上面笑成一塊。月娘道:「六姐你在上頭笑不打緊,只怕一時滑倒,不是耍處。」說著,不想那畫板滑,又是高底鞋,跐不牢,只聽得滑浪一聲把金蓮擦下來,早是扶住架子不曾跌著,險些沒把玉樓也拖下來。月娘道:「我說六姐笑的不好,只當跌下來。」因望李嬌兒眾人說道:「這打鞦韆,最不該笑。笑多了,一定腿軟了,跌下來。咱在家做女兒時,隔壁周台官家花園中扎著一座鞦韆。也是三月佳節,一日他家周小姐和俺一般三四個女孩兒,都打鞦韆耍子,也是這等笑的不了,把周小姐滑下來,騎在畫板上,把身子喜抓去了。落後嫁與人家,被人家說不是女兒,休逐來家,今後打鞦韆,先要忌笑。」金蓮道:「孟三兒不濟,等我和李大姐打個立鞦韆。」月娘道:「你兩個仔細打。」卻教玉簫、春梅在旁推送。才待打時,只見陳敬濟自外來,說道:「你每在這裡打鞦韆哩。」月娘道:「姐夫來的正好,且來替你二位娘送送兒。丫頭每氣力少。」這敬濟老和尚不撞鐘——得不的一聲,於是撥步撩衣,向前說:「等我送二位娘。」先把金蓮裙子帶住,說道:「五娘站牢,兒子送也。」那鞦韆飛在半空中,猶若飛仙相似。李瓶兒見鞦韆起去了,唬的上面怪叫道:「不好了,姐夫你也來送我送兒。」敬濟道:「你老人家到且性急,也等我慢慢兒的打發將來。這裡叫,那裡叫,把兒子手腳都弄慌了。」於是把李瓶兒裙子掀起,露著他大紅底衣,推了一把。李瓶兒道:「姐夫,慢慢著些!我腿軟了!」敬濟道:「你老人家原來吃不得緊酒。」金蓮又說:「李大姐,把我裙子又兜住了。」兩個打到半中腰裡,都下來了。卻是春梅和西門大姐兩個打了一回。然後,教玉簫和蕙蓮兩個打立鞦韆。這蕙蓮手輓彩繩,身子站的直屢屢的,腳跐定下邊畫板,也不用人推送,那鞦韆飛在半天雲里,然後忽地飛將下來,端的卻是飛仙一般,甚可人愛。月娘看見,對玉樓、李瓶兒說:「你看媳婦子,他倒會打。」這裡月娘眾人打鞦韆不題。

話分兩頭。卻表來旺兒往杭州織造蔡太師生辰衣服回來,押著許多馱垛箱籠船上,先走來家。到門首,下了頭口,收卸了行李,進到後邊。只見雪娥正在堂屋門首,作了揖。那雪娥滿面微笑,說道:「好呀,你來家了。路上風霜,多有辛苦!幾時沒見,吃得黑胖了。」來旺因問:「爹娘在那裡?」雪娥道:「你爹今日被應二眾人,邀去門外耍子去了。你大娘和大姐,都在花園中打鞦韆哩。」來旺兒道:「啊呀,打他則甚?」雪娥便倒了一盞茶與他吃,因問:「媳婦子在竈上,怎的不見?」那雪娥冷笑了一聲,說道:「你的媳婦子,如今還是那時的媳婦兒哩?好不大了!他每日只跟著他娘每夥兒里下棋,撾子兒,抹牌頑耍。他肯在竈上做活哩!」正說著,小玉走到花園中,報與月娘。月娘自前邊走來,來旺兒向前磕了頭,立在旁邊。問了些路上往回的話,月娘賞了兩瓶酒。吃一回,他媳婦宋蕙蓮來到。月娘道:「也罷,你辛苦了,且往房裡洗洗頭面,歇宿歇宿去。等你爹來,好見你爹回話。」那來旺兒便歸房裡。蕙蓮先付鑰匙開了門,又舀些水與他洗臉攤塵,收拾褡褳去,說道:「賊黑囚,幾時沒見,便吃得這等肥肥的。」又替他換了衣裳,安排飯食與他吃。睡了一覺起來,已是日西時分。

西門慶來家,來旺兒走到跟前參見,說道:「杭州織造蔡太師生辰的尺頭並家中衣服,俱已完備,打成包裹,裝了四箱,搭在官船上來家,只少雇夫過稅。」西門慶滿心歡喜,與了他趕腳銀兩,明日早裝載進城。又賞銀五兩,房中盤纏;又教他管買辦東西。這來旺兒私已帶了些人事,悄悄送了孫雪娥兩方綾汗巾,兩隻裝花膝褲,四匣杭州粉,二十個胭脂。雪娥背地告訴來旺兒說:「自從你去了四個月,你媳婦怎的和西門慶勾搭,玉簫怎的做牽頭,金蓮屋裡怎的做窩窠。先在山子底下,落後在屋裡,成日明睡到夜,夜睡到明。與他的衣服、首飾、花翠、銀錢,大包帶在身邊。使小廝在門首買東西,見一日也使二三錢銀子。」來旺道:「怪道箱子里放著衣服、首飾!我問他,他說娘與他的。」雪娥道:「那娘與他?到是爺與他的哩!」這來旺兒遂聽記在心。

到晚夕,吃了幾鐘酒,歸到房中。常言酒發頓腹之言,因開箱子,看見一匹藍緞子,甚是花樣奇異,便問老婆:「是那裡的緞子?誰人與你的?趁上實說。」老婆不知就裡,故意笑著,回道:「怪賊囚,問怎的?此是後邊見我沒個襖兒,與了這匹緞子,放在箱中,沒工夫做。端的誰肯與我?」來旺兒罵道:「賊淫婦!還搗鬼哩!端的是那個與你的?」又問:「這些首飾是那裡的?」婦人道:「呸!怪囚根子,那個沒個娘老子,就是石頭罅剌兒里迸出來,也有個窩巢兒,為人就沒個親戚六眷?此是我姨娘家借來的釵梳。是誰與我的!」被來旺兒一拳,險不打了一交,說:「賊淫婦,還說嘴哩!有人親看見你和那沒人倫的豬狗有首尾!玉簫丫頭怎的牽頭,送緞子與你,在前邊花園內兩個乾,落後弔在潘家那淫婦屋裡明乾,成日㒲的不值了。賊淫婦,你還要我手裡吊子曰兒。」那婦人便大哭起來,說道:「賊不逢好死的囚根子!你做甚麼來家打我?我乾壞了你甚麼事來?你恁是言不是語,丟塊磚瓦兒也要個下落。是那個嚼舌根的,沒空生有,調唆你來欺負老娘?我老娘不是那沒根基的貨!教人就欺負死,也揀個乾凈地方。你問聲兒,宋家的丫頭,若把腳略趄兒,把『宋』字兒倒過來!你這賊囚根子,得不個風兒就雨兒。萬物也要個實。人教你殺那個人,你就殺那個人?」幾句說的來旺兒不言語了。婦人又道:「這匹藍緞子,越發我和你說了罷,也是去年十一月里三娘生日,娘見我上穿著紫襖,下邊借了玉簫的裙子穿著,說道:『媳婦子怪剌剌的,甚麼樣子?』才與了我這匹緞子。誰得閑做他?那個是不知道!就纂我恁一遍舌頭。你錯認了老娘,老娘不是個饒人的。明日我咒罵個樣兒與他聽。破著我一條性命,自恁尋不著主兒哩。」來旺兒道:「你既沒此事,平白和人合甚氣?快些打鋪我睡。」這婦人一面把鋪伸下,說道:「怪倒路的囚根子,吃了那黃湯,挺你那覺!平白惹老娘罵。」把來旺掠翻在炕上,鼾聲如雷。看官聽說:但凡世上養漢的婆娘,饒他男子漢十八分精細,吃他幾句左話兒右說,十個九個都著了道兒。正是:
\begin{quote}
東凈里磚兒——又臭又硬。
\end{quote}

這宋蕙蓮窩盤住來旺兒,過了一宿。到次日,往後邊問玉簫,誰人透露此事,終莫知其所由,只顧海罵。一日,月娘使小玉叫雪娥,一地裡尋不著。走到前邊,只見雪娥從來旺兒房裡出來,只猜和他媳婦說話,不想走到廚下,蕙蓮又在裡面切肉,良久,西門慶前邊陪著喬大戶說話,只為揚州鹽商王四峰,被按撫使送監在獄中,許銀二千兩,央西門慶對蔡太師討人情釋放。剛打發大戶去了,西門慶叫來旺,來旺從他屋裡跑出來。正是:
\begin{quote}
雪隱鷺鶯飛始見,柳藏鸚鵡語方知。
\end{quote}

以此都知雪娥與來旺兒有尾首。

一日,來旺兒吃醉了,和一般家人小廝在前邊恨罵西門慶,說怎的我不在家,使玉簫丫頭拿一匹藍緞子,在房裡哄我老婆。把他弔在花園姦耍,後來潘金蓮怎的做窩主:「由他,只休要撞到我手裡。我教他白刀子進去,紅刀子出來。好不好,把潘家那淫婦也殺了,也只是個死。你看我說出來做的出來。潘家那淫婦,想著他在家擺死了他漢子武大,他小叔武鬆來告狀,多虧了誰替他上東京打點,把武鬆墊發充軍去了?今日兩腳踏住平川路,落得他受用,還挑撥我的老婆養漢。我的仇恨,與他結的有天來大。常言道,一不做,二不休,到跟前再說話。破著一命剮,便把皇帝打!」這來旺兒自知路上說話,不知草里有人,不想被同行家人來興兒聽見。這來興兒在家,西門慶原派他買辦食用撰錢過日,只因與來旺媳婦勾搭,把買辦奪了,卻教來旺兒管領。來興兒就與來旺不睦,聽見發此言語,就悄悄走來潘金蓮房裡告訴。

金蓮正和孟玉樓一處坐的,只見來興兒掀帘子進來,金蓮便問來興兒:「你來有甚事?你爹今日往誰家吃酒去了?」來興道:「今日俺爹和應二爹往門外送殯去了。適有一件事,告訴老人家,只放在心裡,休說是小的來說。」金蓮道:「你有甚事,只顧說,不妨事!」來興兒道:「別無甚事,叵耐來旺兒,昨日不知那裡吃的醉稀稀的,在前邊大吆小喝,指豬罵狗,罵了一日。又邏著小的廝打,小的走來一邊不理,他對著家中大小,又罵爹和五娘。」潘金蓮就問:「賊囚根子,罵我怎的?」來興說:「小的不敢說。三娘在這裡,也不是別人。那廝說爹怎的打發他不在家,耍了他的老婆,說五娘怎的做窩主,賺他老婆在房裡和爹兩個明睡到夜,夜睡到明。他打下刀子,要殺爹和五娘,白刀子進去,紅刀子出來。又說,五娘那咱在家,毒藥擺殺了親夫,多虧了他上東京去打點,救了五娘一命。說五娘恩將仇報,挑撥他老婆養漢。小的穿青衣抱黑住,先來告訴五娘說聲,早晚休吃那廝暗算。」玉樓聽了,如提在冷水盆內一般,吃了一驚。這金蓮不聽便罷,聽了,粉面通紅,銀牙咬碎,罵道:「這犯死的奴才!我與他往日無冤近日無讎,他主子要了他的老婆,他怎的纏我?我若教這奴才在西門慶家,永不算老婆!怎的我虧他救活了性命?」因吩咐來興兒:「你且去,等你爹來家問你時,你也只照恁般說。」來興兒說:「五娘說那裡話!小的又不賴他,有一句說一句。隨爹怎的問,也只是這等說。」說畢,往前邊去了。

玉樓便問金蓮:「真個他爹和這媳婦子有?」金蓮道:「你問那沒廉恥的貨!甚的好老婆,也不枉了教奴才這般挾制了。在人家使過了的奴才淫婦,當初在蔡通判家,和大婆作弊養漢,壞了事,才打發出來,嫁了蔣聰。豈止見過一個漢子兒?有一拿小米數兒,甚麼事兒不知道!賊強人瞞神嚇鬼,使玉簫送緞子兒與他做襖兒穿。一冬里,我要告訴你,沒告訴你。那一日,大姐姐往喬大戶家吃酒,咱每都不在前邊下棋?只見丫頭說他爹來家,咱每不散了?落後我走到後邊儀門首,見小玉立在穿廊下,我問他,小玉望著我搖手兒。我剛走到花園前,只見玉簫那狗肉在角門首站立,原來替他觀風。我還不知,教我徑往花園裡走。玉簫攔著我,不教我進去,說爹在裡面。教我罵了兩句。我到疑影和他有些甚麼查子帳,不想走到裡面,他和媳婦子在山洞里乾營生。媳婦子見我進去,把臉飛紅的走出來了。他爹見了我,訕訕的,吃我罵了兩句沒廉恥。落後媳婦子走到屋裡,打旋磨跪著我,教我休對他娘說。落後正月里,他爹要把淫婦安托在我屋裡過一夜兒,吃我和春梅折了兩句,再幾時容他傍個影兒!賊萬殺的奴才,沒的把我扯在裡頭。好嬌態的奴才淫婦,我肯容他在那屋裡頭弄硶兒?就是我罷了,俺春梅那小肉兒,他也不肯容他。」玉樓道:「嗔道賊臭肉在那裡坐著,見了俺每意意似似,待起不起的,誰知原來背地有這本帳!論起來,他爹也不該要他。那裡尋不出老婆來,教奴才在外邊倡揚,甚麼樣子?」金蓮道:「左右的皮靴兒沒番正,你要奴才老婆,奴才暗地裡偷你的小娘子,彼此換著做!賊小婦奴才,千也嘴頭子嚼說人,萬也嚼說,今日打了嘴,也不說的!」玉樓向金蓮道:「這椿事,咱對他爹說好,不說好?大姐姐又不管。倘忽那廝真個安心,咱每不言語,他爹又不知道,一時遭了他手怎了?六姐,你還該說說。」金蓮道:「我若是饒了這奴才,除非是他㒲出我來。」正是:
\begin{quote}
平生不作皺眉事,世上應無切齒人。
\end{quote}

西門慶至晚來家,只見金蓮在房中雲鬟不整,睡搵香腮,哭的眼壞壞的。問其所以,遂把來旺兒醉酒發言,要殺主之事訴說一遍:「見有來興兒親自聽見,思想起來,你背地圖他老婆,他便背地要你家小娘子。你的皮靴兒沒番正。那廝殺你便該當,與我何干?連我一例也要殺!趁早不為之計,夜頭早晚,人無後眼,只怕暗遭他毒手。」西門慶因問:「誰和那廝有首尾?」金蓮道:「你休來問我,只問小玉便知。」又說:「這奴才欺負我,不是一遭兒了。說我當初怎的用藥擺殺漢子,你娶了我來,虧他尋人情搭救我性命來。在外邊對人揭條。早是奴沒生下兒沒長下女,若是生下兒女,教賊奴才揭條著好聽?敢說:『你家娘當初在家不得地時,也虧我尋人情救了他性命。』恁說在你臉上也無光了!你便沒羞恥,我卻成不的,要這命做甚麼?」西門慶聽了婦人之言,走到前邊,叫將來興兒到無人處,問他始末緣由。這小廝一五一十說了一遍。又走到後邊,摘問了小玉口詞,與金蓮所說無差:委的某日,親眼看見雪娥從來旺兒屋裡出來,他媳婦兒不在屋裡,的有此事。這西門慶心中大怒,把孫雪娥打了一頓,被月娘再三勸了,拘了他頭面衣服,只教他伴著家人媳婦上竈,不許他見人。此事表過不題。

西門慶在後邊,因使玉簫叫了宋蕙蓮,背地親自問他。這婆娘便道:「啊呀,爹,你老人家沒的說,他是沒有這個話。我就替他賭了大誓。他酒便吃兩鐘,敢恁七個頭八個膽,背地裡罵爹?又吃紂王水土,又說紂王無道!他靠那裡過日子?爹,你不要聽人言語。我且問爹,聽見誰說這個話來?」那西門慶被婆娘一席話兒,閉口無言。問的急了,說:「是來興兒告訴我說的。」蕙蓮道:「來興兒因爹叫俺這一個買辦,說俺每奪了他的,不得賺些錢使,結下這仇恨兒,平空拿這血口噴他,爹就信了。他有這個欺心的事,我也不饒他。爹你依我,不要教他在家裡,與他幾兩銀子本錢,教他信信脫脫,遠離他鄉,做買賣去。他出去了,早晚爹和我說句話兒也方便些。」西門慶聽了滿心歡喜,說道:「我的兒,說的是。我有心要叫他上東京,與鹽商王四峰央蔡太師人情,回來,還要押送生辰擔去,只因他才從杭州來家,不好又使他的,打帳叫來保去。既你這樣說,我明日打發他去便了。回來,我教他領一千兩銀子,同主管往杭州販買綢絹絲線做買賣。你意下如何?」老婆心中大喜,說道:「爹若這等才好。」正說著,西門慶見無人,就摟他過來親嘴。婆娘忙遞舌頭在他口裡,兩個咂做一處。婦人道:「爹,你許我編鬏髻,怎的還不替我編?恁時候不戴到幾時戴?只教我成日戴這頭髮殼子兒?」西門慶道:「不打緊,到明日將八兩銀子,往銀匠家替你拔絲去。」西門慶又道:「怕你大娘問,怎生回答?」婦人道:「不打緊,我自有話打發他,只說問我姨娘家借來戴戴,怕怎的?」當下二人說了一回話,各自分散了。

到了次日,西門慶在廳上坐著,叫過來旺兒來:「你收拾衣服行李,趕明日三月二十八日起身,往東京央蔡太師人情。回來,我還打發你杭州做買賣去。」這來旺心中大喜,應諾下來,回房收拾行李,在外買人事。來興兒打聽得知,就來告報金蓮知道。金蓮打聽西門慶在花園捲棚內,走到那裡,不見西門慶,只見陳敬濟在那裡封禮物。金蓮便道:「你爹在那裡?你封的是甚麼?」敬濟道:「爹剛纔在這裡,往大娘那邊兌鹽商王四峰銀子去了。我封的是往東京央蔡太師的禮。」金蓮問:「打發誰去?」敬濟道:「我聽見昨日爹吩咐來旺兒去。」這金蓮才待下臺基,往花園那條路上走,正撞見西門慶拿了銀子來。叫到屋裡,問他:「明日打發誰往東京去?」西門慶道:「來旺兒和吳主管二人同去。因有鹽商王四峰一千幹事的銀兩,以此多著兩個去。」婦人道:「隨你心下,我說的話兒你不依,到聽那奴才淫婦一面兒言語。他隨問怎的,只護他的漢子。那奴才有話在先,不是一日兒了。左右破著老婆丟與你,坑了你這銀子,拐的往那頭裡停停脫脫去了,看哥哥兩眼兒空哩。你的白丟了罷了,難為人家一千兩銀子,不怕你不賠他。我說在你心裡,也隨你。老婆無故只是為他。不爭你貪他這老婆,你留他在家裡也不好,你就打發他出去做買賣也不好。你留他在家裡,早晚沒這些眼防範他。你打發他外邊去,他使了你本錢,頭一件你先說不得他。你若要他這奴才老婆,不如先把奴才打發他離門離戶。常言道:剪草不除根,萌芽依舊生;剪草若除根,萌芽再不生。就是你也不耽心,老婆他也死心塌地。」一席話兒,說得西門慶如醉方醒。正是:
\begin{quote}
數語撥開君子路,片言提醒夢中人。
\end{quote}
