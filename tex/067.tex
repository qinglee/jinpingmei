
\chapter{西門慶書房賞雪 李瓶兒夢訴幽情}

詞曰:
\begin{quote}
朔風天,瓊瑤地。凍色連波,波上寒煙砌。山隱彤云云接水,衰草無情,想在彤雲內。
黯香魂,追苦意。夜夜除非,好夢留人睡。殘月高樓休獨倚,酒入愁腸,化作相思淚。
\end{quote}

話說西門慶歸後邊,辛苦的人,直睡至次日日高還未起來。有來興兒進來說:「搭彩匠外邊伺候,請問拆棚。」西門慶罵了來興兒幾句,說:「拆棚教他拆就是了,只顧問怎的!」搭彩匠一面卸下席繩松條,送到對門房子里堆放不題。玉簫進房說:「天氣好不陰的重。」西門慶令他向暖炕上取衣裳穿,要起來。月娘便說:「你昨日辛苦了一夜,天陰,大睡回兒也好。慌的老早爬起去做甚麼?就是今日不往衙門裡去也罷了。」西門慶道:「我不往衙門裡去,只怕翟親家那人來討書。」月娘道:「既是恁說,你起去,我去叫丫鬟熬下粥等你吃。」西門慶也不梳頭洗面,披著絨衣,戴著氈巾,徑走到花園裡書房中。

原來自從書童去了,西門慶就委王經管花園書房,春鴻便收拾大廳前書房。冬月間,西門慶只在藏春閣書房中坐。那裡燒下地爐暖炕,地平上又放著黃銅火盆,放下油單絹暖簾來。明間內擺著夾枝桃,各色菊花,清清瘦竹,翠翠幽蘭,裡面筆硯瓶梅,琴書瀟灑。西門慶進來,王經連忙向流金小篆炷爇龍涎。西門慶使王經:「你去叫來安兒請你應二爹去。」王經出來吩咐來安兒請去了。只見平安走來對王經說:「小周兒在外邊伺候。」王經走入書房對西門慶說了,西門慶叫進小周兒來,磕了頭,說道:「你來得好,且與我篦篦頭,捏捏身上。」因說:「你怎一向不來?」小周兒道:「小的見六娘沒了,忙,沒曾來。」西門慶於是坐在一張醉翁椅上,打開頭髮教他整理梳篦。只見來安兒請的應伯爵來了,頭戴氈帽,身穿綠絨襖子,腳穿一雙舊皂靴棕套,掀帘子進來唱喏。西門慶正篦頭,說道:「不消聲喏,請坐。」伯爵拉過一張椅子來,就著火盆坐下。西門慶道:「你今日如何這般打扮?」伯爵道:「你不知,外邊飄雪花兒哩,好不寒冷。昨日家去,雞也叫了,今日白爬不起來。不是大官兒去叫,我還睡哩。哥,你好漢,還起的早。若是我,成不的。」西門慶道:「早是你看著,我怎得個心閑!自從發送他出去了,又亂著接黃太尉,念經,直到如今。今日房下說:『你辛苦了,大睡回起去。』我又記掛著翟親家人來討回書,又看著拆棚,二十四日又要打發韓伙計和小價起身。喪事費勞了人家,親朋罷了,士大夫官員,你不上門謝謝孝,禮也過不去。」伯爵道:「正是,我愁著哥謝孝這一節。少不的只摘撥謝幾家要緊的,胡亂也罷了。其餘相厚的,若會見,告過就是了。誰不知你府上事多,彼此心照罷。」

正說著,只見畫童兒拿了兩盞酥油白糖熬的牛奶子。伯爵取過一盞,拿在手內,見白瀲瀲鵝脂一般酥油飄浮在盞內,說道:「好東西,滾熱!」呷在口裡,香甜美味,那消氣力,幾口就喝沒了。西門慶直待篦了頭,又教小周兒替他取耳,把奶子放在桌上,只顧不吃。伯爵道:「哥且吃些不是?可惜放冷了。象你清晨吃恁一盞兒,倒也滋補身子。」西門慶道:「我且不吃,你吃了,停會我吃粥罷。」那伯爵得不的一聲,拿在手中,又一吸而盡。西門慶取畢耳,又叫小周兒拿木滾子滾身上,行按摩導引之術。伯爵問道:「哥滾著身子,也通泰自在麼?」西門慶道:「不瞞你說,象我晚夕身上常發酸起來,腰背疼痛,不著這般按捏,通了不得!」伯爵道:「你這胖大身子,日逐吃了這等厚味,豈無痰火!」西門慶道:「任後溪常說:『老先生雖故身體魁偉,而虛之太極。』送了我一罐兒百補延齡丹,說是林真人合與聖上吃的,教我用人乳常清晨服。我這兩日心上亂,也還不曾吃。你們只說我身邊人多,終日有此事,自從他死了,誰有甚麼心緒理論此事!」

正說著,只見韓道國進來,作揖坐下,說:「剛纔各家都來會了,船已雇下,準在二十四日起身。」西門慶吩咐:「甘伙計攢下帳目,兌了銀子,明日打包。」因問:「兩邊鋪子里賣下多少銀兩?」韓道國說:「共湊六千餘兩。」西門慶道:「兌二千兩一包,著崔本往湖州買綢子去。那四千兩,你與來保往松江販布,過年趕頭水船來。你每人先拿五兩銀子,家中收拾行李去。」韓道國道:「又一件:小人身從鄆王府,要正身上直,不納官錢如何處?」西門慶道:「怎的不納官錢?象來保一般也是鄆王差事,他每月只納三錢銀子。」韓道國道:「保官兒那個,虧了太師老爺那邊文書上註過去,便不敢纏擾。小人乃是祖役,還要勾當餘丁。」西門慶道:「既是如此,你寫個揭帖,我央任後溪到府中替你和王奉承說,把你名字註銷,常遠納官錢罷。你每月只委人打米就是了。」韓伙計作揖謝了。伯爵道:「哥,你替他處了這件事,他就去也放心。」少頃,小周滾畢身上,西門慶往後邊梳頭去了,吩咐打發小周兒吃點心。

良久,西門慶出來,頭戴白絨忠靖冠,身披絨氅,賞了小周三錢銀子。又使王經:「請你溫師父來。」不一時,溫秀才峨冠博帶而至。敘禮已畢,左右放桌兒,拿粥來,伯爵與溫秀才上坐,西門慶關席,韓道國打橫。西門慶吩咐來安兒:「再取一盞粥、一雙筷兒,請姐夫來吃粥。」不一時,陳敬濟來到,頭戴孝巾,身穿白綢道袍,與伯爵等作揖,打橫坐下。須臾吃了粥,收下家火去,韓道國起身去了。西門慶因問溫秀才:「書寫了不曾?」溫秀才道:「學生已寫稿在此,與老先生看過,方可謄真。」一面袖中取出,遞與西門慶觀看。其書曰:
\begin{quote}
寓清河眷生西門慶端肅書復大碩德柱國雲峰老親丈大人先生臺下:自從京邸邂逅,不覺違越光儀,倏忽半載。生不幸閨人不祿,特蒙親家遠致賻儀,兼領悔教,足見為我之深且厚也。感刻無任,而終身不能忘矣。但恐一時官守責成有所疏陋之處,企仰門牆有負薦拔耳,又賴在老爺鈞前常為錦覆。則生始終蒙恩之處,皆親家所賜也。今因便鴻謹候起居,不勝馳戀,伏惟照亮,不宣。外具揚州縐紗汗巾十方、色綾汗巾十方、揀金挑牙二十付、烏金酒鐘十個,少將遠意,希笑納。
\end{quote}

西門慶看畢,即令陳敬濟書房內取出人事來,同溫秀才封了,將書謄寫錦箋,彌封停當,印了圖書。另外又封五兩白銀與下書人王玉,不在話下。

一回見雪下的大了,西門慶留下溫秀才在書房中賞雪。揩抹桌兒,拿上案酒來。只見有人在暖簾外探頭兒,西門慶問是誰,王經說:「是鄭春。」西門慶叫他進來。那鄭春手內拿著兩個盒兒,舉的高高的,跪在當面,上頭又擱著個小描金方盒兒,西門慶問是甚麼,鄭春道:「小的姐姐月姐,知道昨日爹與六娘念經辛苦了,沒甚麼,送這兩盒兒茶食兒來,與爹賞人。」揭開,一盒果餡頂皮酥、一盒酥油泡螺兒。鄭春道:「此是月姐親手揀的。知道爹好吃此物,敬來孝順爹。」西門慶道:「昨日多謝你家送茶,今日你月姐費心又送這個來。」伯爵道:「好呀!拿過來,我正要嘗嘗!死了我一個女兒會揀泡螺兒,如今又是一個女兒會揀了。」先捏了一個放在口內,又拈了一個遞與溫秀才,說道:「老先兒,你也嘗嘗。吃了牙老重生,抽胎換骨。眼見希奇物,勝活十年人。」溫秀才呷在口內,入口而化,說道:「此物出於西域,非人間可有。沃肺融心,實上方之佳味。」西門慶又問:「那小盒兒內是甚麼?」鄭春悄悄跪在西門慶跟前,遞上盒兒,說:「此是月姐捎與爹的物事。」西門慶把盒子放在膝蓋兒上,揭開才待觀看,早被伯爵一手撾過去,打開是一方回紋錦同心方勝桃紅綾汗巾兒,裡面裹著一包親口嗑的瓜仁兒。伯爵把汗巾兒掠與西門慶,將瓜仁兩把喃在口裡都吃了。比及西門慶用手奪時,只剩下沒多些兒,便罵道:「怪狗才,你害饞癆饞痞!留些兒與我見見兒,也是人心。」伯爵道:「我女兒送來,不孝順我,再孝順誰?我兒,你尋常吃的夠了。」西門慶道:「溫先兒在此,我不好罵出來,你這狗才,忒不象模樣!」一面把汗巾收入袖中,吩咐王經把盒兒掇到後邊去。

不一時,杯盤羅列,篩上酒來。才吃了一巡酒,玳安兒來說:「李智、黃四關了銀子,送銀子來了。」西門慶問多少,玳安道:「他說一千兩,餘者再一限送來。」伯爵道:「你看這兩個天殺的,他連我也瞞了不對我說。嗔道他昨日你這裡念經他也不來,原來往東平府關銀子去了。你今收了,也少要發銀子出去了。這兩個光棍,他攬的人家債多了,只怕往後後手不接。昨日,北邊徐內相發恨,要親往東平府自家抬銀子去。只怕他老牛箍嘴箍了去,卻不難為哥的本錢!」西門慶道:「我不怕他。我不管甚麼徐內相李內相,好不好把他小廝提在監里坐著,不怕他不與我銀子。」一面教陳敬濟:「你拿天平出去收兌了他的就是了。我不出去罷。」

良久,陳敬濟走來回話說:「銀子已兌足一千兩,交入後邊,大娘收了。黃四說,還要請爹出去說句話兒。」西門慶道:「你只說我陪著人坐著哩。左右他只要搗合同,教他過了二十四日來罷。」敬濟道:「不是。他說有樁事兒要央煩爹。」西門慶道:「甚麼事?等我出去。」一面走到廳上,那黃四磕頭起來,說:「銀子一千兩,姐夫收了。餘者下單我還。小人有一樁事兒央煩老爹。」說著磕在地下哭了。西門慶拉起來道:「端的有甚麼事,你說來。」黃四道:「小的外父孫清,搭了個伙計馮二,在東昌府販綿花。不想馮二有個兒子馮淮,不守本分,要便鎖了門出去宿娼。那日把綿花不見了兩大包,被小人丈人說了兩句,馮二將他兒子打了兩下。他兒子就和俺小舅子孫文相廝打起來,把孫文相牙打落了一個,他亦把頭磕傷。被客夥中解勸開了。不想他兒子到家,遲了半月,破傷風身死。他丈人是河西有名土豪白五,綽號白千金,專一與強盜做窩主,教唆馮二,具狀在巡按衙門朦朧告下來,批雷兵備老爹問。雷老爹又伺候皇船,不得閑,轉委本府童推官問。白家在童推官處使了錢,教鄰見人供狀,說小人丈人在旁喝聲來。如今童推官行牌來提俺丈人。望乞老爹千萬垂憐,討封書對雷老爹說,寧可監幾日,抽上文書去,還見雷老爹問,就有生路了。他兩人廝打,委的不管小人丈人事,又系歇後身死,出於保辜限外。先是他父馮二打來,何必獨賴孫文相一人身上?」西門慶看了說帖,寫著:「東昌府見監犯人孫清、孫文相,乞青目。」因說:「雷兵備前日在我這裡吃酒,我只會了一面,又不甚相熟,我怎好寫書與他?」黃四就跪下哭哭啼啼哀告說:「老爹若不可憐見,小的丈人子父兩個就都是死數了。如今隨孫文相出去罷了,只是分豁小人外父出來,就是老爹莫大之恩。小人外父今年六十歲,家下無人,冬寒時月再放在監里,就死罷了。」西門慶沉吟良久,說:「也罷,我轉央鈔關錢老爹和他說說去——與他是同年,都是壬辰進士。」黃四又磕下頭去,向袖中取出「一百石白米」帖兒遞與西門慶,腰裡就解兩封銀子來。西門慶不接,說道:「我那裡要你這行錢!」黃四道:「老爹不稀罕,謝錢老爹也是一般。」西門慶道:「不打緊,事成我買禮謝他。」

正說著,只見應伯爵從角門首出來,說:「哥,休替黃四哥說人情。他閑時不燒香,忙時抱佛腿。昨日哥這裡念經,連茶兒也不送,也不來走走兒,今日還來說人情!」那黃四便與伯爵唱喏,說道:「好二叔,你老人家殺人哩!我因這件事,整走了這半月,誰得閑來?昨日又去府里領這銀子,今日一來交銀子,就央說此事,救俺丈人。老爹再三不肯收這禮物,還是不下顧小人。」伯爵看見一百兩雪花官銀放在面前,因問:「哥,你替他去說不說?」西門慶道:「我與雷兵備不熟,如今要轉央鈔關錢主政替他說去。到明日,我買分禮謝老錢就是了,又收他禮做甚麼?」伯爵道:「哥,你這等就不是了。難道他來說人情,哥你倒陪出禮去謝人?也無此道理。你不收,恰似嫌少的一般。你依我收下。雖你不稀罕,明日謝錢公也是一般。黃四哥在這裡聽著:看你外父和你小舅子造化,這一回求了書去,難得兩個都沒事出來。你老爹他恆是不稀罕你錢,你在院里老實大大擺一席酒,請俺們耍一日就是了。」黃四道:「二叔,你老人家費心,小人擺酒不消說,還叫俺丈人買禮來,磕頭酬謝你老人家。不瞞說,我為他爺兒兩個這一場事,晝夜替他走跳,還尋不出個門路來。老爹再不可憐怎了!」伯爵道:「傻瓜,你摟著他女兒,你不替他上緊誰上緊?」黃四道:「房下在家只是哭。」西門慶被伯爵說著,把禮帖收了,說禮物還令他拿回去。黃四道:「你老人家沒見好大事,這般多計較!」就往外走。伯爵道:「你過來,我和你說:你書幾時要?」黃四道:「如今緊等著救命,望老爹今日寫了書,差下人,明早我使小兒同去走遭。不知差那位大官兒去,我會他會。」西門慶道:「我就替你寫書。」因叫過玳安來吩咐:「你明日就同黃大官一路去。」

那黃四見了玳安,辭西門慶出門。走到門首,問玳安要盛銀子的褡褳。玳安進入後邊,月娘房裡正與玉簫、小玉裁衣裳,見玳安站著等褡褳,玉簫道:「使著手,不得閑謄。教他明日來與他就是了。」玳安道:「黃四等緊著明日早起身東昌府去,不得來了,你謄謄與他罷。」月娘便說:「你拿與他就是了,只教人家等著。」玉簫道:「銀子還在床地平上掠著不是?」走到裡間,把銀子往床上只一倒,掠出褡褳來,說:「拿了去!怪囚根子,那個吃了他這條褡褳,只顧立叮螞蝗的要!」玳安道:「人家不要,那個好來取的!」於是拿了出去,走到儀門首,還抖出三兩一塊麻姑頭銀子來。原來紙包破了,怎禁玉簫使性子那一倒,漏下一塊在褡褳底內。玳安道:「且喜得我拾個白財。」於是褪入袖中。到前邊遞與黃四,約會下明早起身。

且說西門慶回到書房中,即時教溫秀才修了書,付與玳安不題。一面覷那門外下雪,紛紛揚揚,猶如風飄柳絮,亂舞梨花相似。西門慶另打開一壇雙料麻姑酒,教春鴻用布甑篩上來,鄭春在旁彈箏低唱,西門慶令他唱一套「柳底風微」。正唱著,只見琴童進來說:「韓大叔教小的拿了這個帖兒與爹瞧。」西門慶看了,吩咐:「你就拿往門外任醫官家,替他說說去。央他明日到府中承奉處替他說說,註銷差事。」琴童道:「今日晚了,小的明早去罷。」西門慶道:「明早去也罷。」不一時,來安兒用方盒拿了八碗下飯,又是兩大盤玫瑰鵝油燙麵蒸餅,連陳敬濟共四人吃了。西門慶教王經盒盤兒拿兩碗下飯、一盤點心與鄭春吃,又賞了他兩大鐘酒。鄭春跪稟:「小的吃不的。」伯爵道:「傻孩子,冷呵呵的,你爹賞你不吃。你哥他怎的吃來?」鄭春道:「小的哥吃的,小的本吃不的。」伯爵道:「你只吃一鐘罷,那一鐘我教王經替你吃罷。」王經說道:「二爹,小的也吃不的。」伯爵道:「你這傻孩兒,你就替他吃些兒也罷。休說一個大分上,自古長者賜,少者不敢辭。」一面站起來說:「我好歹教你吃這一杯。」那王經捏著鼻子,一吸而飲。西門慶道:「怪狗才,小行貨子他吃不的,只恁奈何他!」還剩下半盞,應伯爵教春鴻替他吃了,就要令他上來唱南曲。西門慶道:「咱每和溫老先兒行個令,飲酒之時教他唱便有趣。」於是教王經取過骰盆兒,「就是溫老先兒先起。」溫秀才道:「學生豈敢僭,還從應老翁來。」因問:「老翁尊號?」伯爵道:「在下號南坡。」西門慶戲道:「老先生你不知,他孤老多,到晚夕桶子掇出來,不敢在左近倒,恐怕街坊人罵,教丫頭直掇到大南首縣倉牆底下那裡潑去,因起號叫做『南潑』。」溫秀才笑道:「此『坡』字不同。那『潑』字乃點水邊之『發』,這『坡』字卻是『土』字旁邊著個『皮』字。」西門慶道:「老先兒倒猜得著,他娘子鎮日著皮子纏著哩。」溫秀才笑道:「豈有此說?」伯爵道:「葵軒,你不知道,他自來有些快傷叔人家。」溫秀才道:「自古言不褻不笑。」伯爵道:「老先兒,誤了咱每行令,只顧和他說甚麼,他快屎口傷人!你就在手,不勞謙遜。」溫秀才道:「擲出幾點,不拘詩詞歌賦,要個『雪』字,就照依點數兒上。說過來,飲一小杯;說不過來,吃一大盞。」溫秀才擲了個幺點,說道:「學生有了:雪殘鸂𪄠亦多時。」推過去,該應伯爵行,擲出個五點來。伯爵想了半日,想不起來,說:「逼我老人家命也!」良久,說道:「可怎的也有了。」說道:「雪裡梅花雪裡開。——好不好?」溫秀才道:「南老說差了,犯了兩個『雪』字,頭上多了一個『雪』字。」伯爵道:「頭上只小雪,後來下大雪來了。」西門慶道:「這狗才,單管胡說。」教王經斟上大鐘,春鴻拍手唱南曲《駐馬聽》:
\begin{quote}
寒夜無茶,走向前村覓店家。這雪輕飄僧舍,密灑歌樓,遙阻歸槎。
江邊乘興探梅花,庭中歡賞燒銀蠟。一望無涯,有似灞橋柳絮滿天飛下。
\end{quote}

伯爵才待拿起酒來吃,只見來安兒後邊拿了幾碟果食,內有一碟酥油泡螺,又一碟黑黑的團兒,用桔葉裹著。伯爵拈將起來,聞著噴鼻香,吃到口猶如飴蜜,細甜美味,不知甚物。西門慶道:「你猜?」伯爵道:「莫非是糖肥皂?」西門慶笑道:「糖肥皂那有這等好吃。」伯爵道:「待要說是梅酥丸,裡面又有核兒。」西門慶道:「狗才過來,我說與你罷,你做夢也夢不著。是昨日小價杭州船上捎來,名喚做衣梅。都是各樣藥料和蜜煉製過,滾在楊梅上,外用薄荷、桔葉包裹,才有這般美味。每日清晨噙一枚在口內,生津補肺,去惡味,煞痰火,解酒克食,比梅酥丸更妙。」伯爵道:「你不說,我怎的曉得。」因說:「溫老先兒,咱再吃個兒。」教王經:「拿張紙兒來,我包兩丸兒,到家捎與你二娘吃。」又拿起泡螺兒來問鄭春:「這泡螺兒果然是你家月姐親手揀的?」鄭春跪下說:「二爹,莫不小的敢說謊?不知月姐費了多少心,只揀了這幾個兒來孝順爹。」伯爵道:「可也虧他,上頭紋溜,就象螺螄兒一般,粉紅、純白兩樣兒。」西門慶道:「我兒,此物不免使我傷心。惟有死了的六娘他會揀,他沒了,如今家中誰會弄他!」伯爵道:「我頭裡不說的,我愁甚麼?死了一個女兒會揀泡螺兒孝順我,如今又鑽出個女兒會揀了。偏你也會尋,尋的都是妙人兒。」西門慶笑的兩眼沒縫兒,趕著伯爵打,說:「你這狗才,單管只胡說。」溫秀才道:「二位老先生可謂厚之至極。」伯爵道:「老先兒你不知,他是你小侄人家。」西門慶道:「我是他家二十年舊孤老。」陳敬濟見二人犯言,就起身走了。那溫秀才只是掩口而笑。

須臾,伯爵飲過大鐘,次該西門慶擲骰兒。於是擲出個七點來,想了半日說:「我說《香羅帶》上一句唱:『東君去意切,梨花似雪。』」伯爵道:「你說差了,此在第九個字上了,且吃一大鐘。」於是流沿兒斟了一銀衢花鐘,放在西門慶面前,教春鴻唱,說道:「我的兒,你肚子里裹棗核解板兒——能有幾句!」春鴻又拍手唱了一個。看看飲酒至昏,掌燭上來。西門慶飲過,伯爵道:「姐夫不在,溫老先生你還該完令。」溫秀才拿起骰兒,擲出個幺點,想了想,見壁上掛著一幅弔屏,泥金書一聯:「風飄弱柳平橋晚;雪點寒梅小院春。」就說了末後一句。伯爵道:「不算,不算,不是你心上發出來的。該吃一大鐘。」春鴻斟上,那溫秀才不勝酒力,坐在椅上只顧打盹,起來告辭。伯爵還要留他,西門慶道:「罷罷!老先兒他斯文人,吃不的。」令畫童兒:「你好好送你溫師父那邊歇去。」溫秀才得不的一聲,作別去了。伯爵道:「今日葵軒不濟,吃了多少酒兒?就醉了。」於是又飲夠多時,伯爵起身說:「地下滑,我也酒夠了。」因說:「哥,明日你早教玳安替他下書去。」西門慶道:「你不見我交與他書,明日早去了。」伯爵掀開帘子,見天陰地下滑,旋要了個燈籠,和鄭春一路去。西門慶又與了鄭春五錢銀子,盒內回了一罐衣梅,捎與他姐姐鄭月兒吃。臨出門,西門慶因戲伯爵:「你哥兒兩個好好去。」伯爵道:「你多說話。父子上山,各人努力。好不好,我如今就和鄭月兒那小淫婦兒答話去。」說著,琴童送出門去了。

西門慶看收了傢伙,扶著來安兒,打燈籠入角門,從潘金蓮門首過,見角門關著,悄悄就往李瓶兒房裡來。彈了彈門,繡春開了門,來安就出去了。西門慶進入明間,見李瓶兒影,就問:「供養了羹飯不曾?」如意兒就出來應道:「剛纔我和姐供養了。」西門慶椅上坐了,迎春拿茶來吃了。西門慶令他解衣帶,如意兒就知他在這房裡歇,連忙收拾床鋪,用湯婆熨的被窩暖洞洞的,打發他歇下。繡春把角門關了,都在明間地平上支著板凳,打鋪睡下。西門慶要茶吃,兩個已知科範,連忙攛掇奶子進去和他睡。老婆脫衣服鑽入被窩內,西門慶乘酒興服了藥,那話上使了托子,老婆仰臥炕上,架起腿來,極力鼓搗,沒高低扇磞,扇磞的老婆舌尖冰冷,淫水溢下,口中呼「達達」不絕。夜靜時分,其聲遠聆數室。西門慶見老婆身上如綿瓜子相似,用一雙胳膊摟著他,令他蹲下身子,在被窩內咂雞巴,老婆無不曲體承奉。西門慶說:「我兒,你原來身體皮肉也和你娘一般白凈,我摟著你,就如和他睡一般。你須用心伏侍我,我看顧你。」老婆道:「爹沒的說,將天比地,折殺奴婢!奴婢男子漢已沒了,爹不嫌醜陋,早晚只看奴婢一眼兒就夠了。」西門慶便問:「你年紀多少?」老婆道:「我今年屬免的,三十一歲了。」西門慶道:「你原來小我一歲。」見他會說話兒,枕上又好風月,心下甚喜。早晨起來,老婆伏侍拿鞋襪,打發梳洗,極盡殷勤,把迎春、繡春打靠後。又問西門慶討蔥白綢子:「做披襖子,與娘穿孝。」西門慶一一許他。就教小廝鋪子里拿三匹蔥白綢來:「你每一家裁一件。」瞞著月娘,背地銀錢、衣服、首飾,甚麼不與他!

次日,潘金蓮就打聽得知,走到後邊對月娘說:「大姐姐,你不說他幾句!賊沒廉恥貨,昨日悄悄鑽到那邊房裡,與老婆歇了一夜。餓眼見瓜皮,甚麼行貨子,好的歹的攬搭下。不明不暗,到明日弄出個孩子來算誰的?又象來旺兒媳婦子,往後教他上頭上臉,甚麼張致!」月娘道:「你們只要栽派教我說,他要了死了的媳婦子,你每背地都做好人兒,只把我合在缸底下。我如今又做傻子哩!你每說只顧和他說,我是不管你這閑帳。」金蓮見月娘這般說,一聲兒不言語,走回房去了。

西門慶早起見天晴了,打發玳安往錢主事家下書去了。往衙門回來,平安兒來稟:「翟爹人來討書。」西門慶打發書與他,因問那人:「你怎的昨日不來取?」那人說:「小的又往巡撫侯爺那裡下書來,耽擱了兩日。」說畢,領書出門。西門慶吃了飯就過對門房子里,看著兌銀、打包、寫書帳。二十四日燒紙,打發韓伙計、崔本並後生榮海、胡秀五人起身往南邊去。寫了一封書捎與苗小湖,就謝他重禮。

看看過了二十五六,西門慶謝畢孝,一日早晨,在上房吃了飯坐的。月娘便說:「這出月初一日,是喬親家長姐生日,咱也還買份禮兒送了去。常言先親後不改,莫非咱家孩兒沒了,就斷禮不送了?」西門慶道:「怎的不送!」於是吩咐來興買四盒禮,又是一套妝花緞子衣服、兩方銷金汗巾、一盒花翠。寫帖兒,叫王經送了去。這西門慶吩咐畢,就往花園藏春閣書房中坐的。只見玳安下了書回來回話,說:「錢老爹見了爹的帖子,隨即寫書差了一吏,同小的和黃四兒子到東昌府兵備道下與雷老爹。雷老爹旋行牌問童推官催文書,連犯人提上去從新問理。連他家兒子孫文相都開出來,只追了十兩燒埋錢,問了個不應罪名,杖七十,罰贖。復又到鈔關上回了錢老爹話,討了回帖,才來了。」西門慶見玳安中用,心中大喜。拆開回帖觀看,原來雷兵備回錢主事帖子都在裡面。上寫道:
\begin{quote}
來諭悉已處分,但馮二已曾責子在先,何況與孫文相忿毆,彼此俱傷,歇後身死,又在保辜限外,問之抵命,難以平允。量追燒埋錢十兩給與馮二,相應發落。謹此回覆。
\end{quote}

下書:「年侍生雷啟元再拜。」

西門慶看了歡喜,因問:「黃四舅子在那裡?」玳安道:「他出來都往家去了。明日同黃四來與爹磕頭。黃四丈人與了小的一兩銀子。」西門慶吩咐置鞋腳穿,玳安磕頭而出。西門慶就歪在床炕上眠著了。王經在桌上小篆內炷了香,悄悄出來了。良久,忽聽有人掀的簾兒響,只見李瓶兒驀地進來,身穿糝紫衫、白絹裙,亂輓烏雲,黃懨懨面容,向床前叫道:「我的哥哥,你在這裡睡哩,奴來見你一面。我被那廝告了一狀,把我監在獄中,血水淋漓,與穢污在一處,整受了這些時苦。昨日蒙你堂上說了人情,減我三等之罪。那廝再三不肯,發恨還要告了來拿你。我待要不來對你說,誠恐你早晚暗遭毒手。我今尋安身之處去也,你須防範他。沒事少要在外吃夜酒,往那去,早早來家。千萬牢記奴言,休要忘了!」說畢,二人抱頭而哭。西門慶便問:「姐姐,你往那去?對我說。」李瓶兒頓脫,撒手卻是南柯一夢。西門慶從睡夢中直哭醒來,看見簾影射入,正當日午,由不的心中痛切。正是:
\begin{quote}
花落土埋香不見,鏡空鸞影夢初醒。
\end{quote}
有詩爲證:
\begin{quote}
殘雪初晴照紙窗,地爐灰燼冷侵床。
個中邂逅相思夢,風撲梅花鬥帳香。
\end{quote}

不想早晨送了喬親家禮,喬大戶娘子使了喬通來送請帖兒,請月娘眾姊妹。小廝說:「爹在書房中睡哩。」都不敢來問。月娘在後邊管待喬通,潘金蓮說:「拿帖兒,等我問他去。」於是驀地推開書房門,見西門慶歪著,他一屁股就坐在旁邊,說:「我的兒,獨自個自言自語,在這裡做甚麼?嗔道不見你,原來在這裡好睡也!」一面說話,一面看著西門慶,因問:「你的眼怎生揉的恁紅紅的?」西門慶道:「想是我控著頭睡來。」金蓮道:「到只象哭的一般。」西門慶道:「怪奴才,我平白怎的哭?」金蓮道:「只怕你一時想起甚心上人兒來是的。」西門慶道:「沒的胡說,有甚心上人、心下人?」金蓮道:「李瓶兒是心上的,奶子是心下的,俺們是心外的人,入不上數。」西門慶道:「怪小淫婦兒,又六說白道起來。」因問:「我和你說正經話——前日李大姐裝槨,你每替他穿了甚麼衣服在身底下來?」金蓮道:「你問怎的?」西門慶道:「不怎的,我問聲兒。」金蓮道:「你問必有緣故。上面穿兩套遍地金緞子衣服,底下是白綾襖、黃綢裙,貼身是紫綾小襖、白絹裙、大紅小衣。」西門慶點了點頭兒。金蓮道:「我做獸醫二十年,猜不著驢肚裡病?你不想他,問他怎的?」西門慶道:「我才方夢見他來。」金蓮道:「夢是心頭想,噴涕鼻子癢。饒他死了,你還這等念他。象俺每都是可不著你心的人,到明日死了,苦惱也沒那人想念!」西門慶向前一手摟過他脖子來,就親個嘴,說:「怪小油嘴,你有這些賊嘴賊舌的。」金蓮道:「我的兒,老娘猜不著你那黃貓黑尾的心兒!」兩個又咂了一回舌頭,自覺甜唾溶心,脂滿香唇,身邊蘭麝襲人。西門慶於是淫心輒起,摟他在懷裡。他便仰靠梳背,露出那話來,叫婦人品簫。婦人真個低垂粉頭,吞吐裹沒,往來鳴咂有聲。西門慶見他頭上戴金赤虎分心,香雲上圍著翠梅花鈿兒,後髩上珠翹錯落,興不可遏。正做到美處,忽見來安兒隔簾說:「應二爹來了。」西門慶道:「請進來。」慌的婦人沒口子叫:「來安兒賊囚,且不要叫他進來,等我出去著。」來安兒道:「進來了,在小院內。」婦人道:「還不去教他躲躲兒!」那來安兒走去,說:「二爹且閃閃兒,有人在屋裡。」這伯爵便走到松牆旁邊,看雪培竹子。王經掀著軟簾,只聽裙子響,金蓮一溜煙後邊走了。正是:
\begin{quote}
雪隱鷺鷥飛始見,柳藏鸚鵡語方知。
\end{quote}

伯爵進來,見西門慶,唱喏坐下。西門慶道:「你連日怎的不來?」伯爵道:「哥,惱的我要不的在這裡。」西門慶問道:「又怎的惱?你告我說。」伯爵道:「緊自家中沒錢,昨日俺房下那個,平白又桶出個孩兒來。白日里還好撾撓,半夜三更,房下又七痛八病。少不得扒起來收拾草紙被褥,叫老娘去。打緊應保又被俺家兄使了往莊子上馱草去了。百忙撾不著個人,我自家打燈籠叫了巷口鄧老娘來。及至進門,養下來了。」西門慶問:「養個甚麼?」伯爵道:「養了個小廝。」西門慶罵道:「傻狗才,生了兒子倒不好,如何反惱?是春花兒那奴才生的?」伯爵笑道:「是你春姨。」西門慶道:「那賊狗掇腿的奴才,誰教你要他來?叫叫老娘還抱怨!」伯爵道:「哥,你不知,冬寒時月,比不的你們有錢的人家,又有偌大前程,生個兒子錦上添花,便喜歡。俺們連自家還多著個影兒哩,要他做甚麼!家中一窩子人口要吃穿,巴劫的魂也沒了。應保逐日該操當他的差事去了,家兄那裡是不管的。大小女便打發出去了,天理在頭上,多虧了哥你。眼見的這第二個孩兒又大了,交年便是十三歲。昨日媒人來討帖兒。我說:『早哩,你且去著。』緊自焦的魂也沒了,猛可半夜又鑽出這個業障來。那黑天摸地,那裡活變錢去?房下見我抱怨,沒奈何,把他一根銀挖兒與了老娘去了。明日洗三,嚷的人家知道了,到滿月拿甚麼使?到那日我也不在家,信信拖拖到那寺院里且住幾日去罷。」西門慶笑道:「你去了,好了和尚來趕熱被窩兒。你這狗才,到底佔小便益兒。」又笑了一回,那應伯爵故意把嘴谷都著不做聲。西門慶道:「我的兒,不要惱,你用多少銀子,對我說,等我與你處。」伯爵道:「有甚多少?」西門慶道:「也夠你攪纏是的。到其間不夠了,又拿衣服當去。」伯爵道:「哥若肯下顧,二十兩銀子就夠了,我寫個符兒在此。費煩的哥多了,不好開口的,也不敢填數兒,隨哥尊意便了。」西門慶也不接他文約,說:「沒的扯淡,朋友家,什麼符兒!」正說著,只見來安兒拿茶進來。西門慶叫小廝:「你放下盞兒,喚王經來。」不一時,王經來到。西門慶吩咐:「你往後邊對你大娘說,我裡間床背閣上,有前日巡按宋老爹擺酒兩封銀子,拿一封來。」王經應諾,不多時拿了銀子來。西門慶就遞與應伯爵,說:「這封五十兩,你都拿了使去。原封未動,你打開看看。」伯爵道:「忒多了。」西門慶道:「多的你收著,眼下你二令愛不大了?你可也替他做些鞋腳衣裳,到滿月也好看。」伯爵道:「哥說的是。」將銀子拆開,都是兩司各府傾就分資,三兩一錠,松紋足色,滿心歡喜,連忙打恭致謝,說道:「哥的盛情,誰肯!真個不收符兒?」西門慶道:「傻孩兒,誰和你一般計較?左右我是你老爺老娘家,不然你但有事就來纏我?這孩子也不是你的孩子,自是咱兩個分養的。實和你說,過了滿月,把春花兒那奴才叫了來,且答應我些時兒,只當利錢不算罷。」伯爵道:「你春姨這兩日瘦的象你娘那樣哩!」兩個戲了一回,伯爵因問:「黃四丈人那事怎樣了?」西門慶說:「錢龍野書到,雷兵備旋行牌提了犯人上去從新問理,把孫文相父子兩個都開出來,只認了十兩燒埋錢。」伯爵道:「造化他了。他就點著燈兒,那裡尋這人情去!你不受他的,乾不受他的。雖然你不稀罕,留送錢大人也好。別要饒了他,教他好歹擺一席大酒,裡邊請俺們坐一坐。你不說,等我和他說。饒了他小舅一個死罪,當別的小可事兒!」這裡說話不題。

且說月娘在上房,只見孟玉樓走來,說他兄弟孟銳:「不久又起身往川廣販雜貨去。今來辭辭他爹,在我屋裡坐著哩。他在那裡?姐姐使個小廝對他說聲兒。」月娘道:「他在花園書房和應二坐著哩。又說請他爹哩,頭裡潘六姐到請的好!喬通送帖兒來,等著討個話兒,到明日咱們好去不去。我便把喬通留下,打發吃茶,長等短等不見來,熬的喬通也去了。半日,只見他從前邊走將來,教我問他:『你對他說了不曾?』他沒的話回,只噦了一聲:『我就忘了。』帖子還袖在袖子里。原來是恁個沒尾巴行貨子!不知前頭乾甚麼營生,那半日才進來,恰好還不曾說。吃我訌了兩句,往前去了。」少頃,來安進來,月娘使他請西門慶,說孟二舅來了。西門慶便起身,留伯爵:「你休去了,我就來。」走到後邊,月娘先把喬家送帖來請說了。西門慶說:「那日只你一人去罷。熱孝在身,莫不一家子都出來!」月娘說:「他孟二舅來辭辭你,一兩日就起身往川廣去。在三姐屋裡坐著哩。」又問:「頭裡你要那封銀子與誰?」西門慶道:「應二哥房裡春花兒,昨晚生了個兒子,問我借幾兩銀子使。告我說,他第二個女兒又大,愁的要不的。」月娘道:「好,好。他恁大年紀,也才見這個孩子,應二嫂不知怎的喜歡哩!到明日,咱也少不的送些粥米兒與他。」西門慶道:「這個不消說。到滿月,不要饒花子,奈何他好歹發帖兒,請你們往他家走走去,就瞧瞧春花兒怎麼模樣。」月娘笑道:「左右和你家一般樣兒,也有鼻兒也有眼兒,莫不差別些兒!」一面使來安請孟二舅來。

不一時,孟玉樓同他兄弟來拜見。敘禮已畢,西門慶陪他敘了回話,讓至前邊書房內與伯爵相見。吩咐小廝看菜兒,放桌兒篩酒上來,三人飲酒。西門慶教再取雙鐘箸:「對門請溫師父陪你二舅坐。」來安不一時回說:「溫師父不在,望倪師父去了。」西門慶說:「請你姐夫來坐坐。」良久,陳敬濟來,與二舅見了禮,打橫坐下。西門慶問:「二舅幾時起身,去多少時?」孟銳道:「出月初二日準起身。定不的年歲,還到荊州買紙,川廣販香蠟,著緊一二年也不止。販畢貨就來家了。此去從河南、陝西、漢州去,回來打水路從峽江、荊州那條路來,往回七八千里地。」伯爵問:「二舅貴庚多少?」孟銳道:「在下虛度二十六歲。」伯爵道:「虧你年小小的,曉的這許多江湖道路,似俺們虛老了,只在家裡坐著。」須臾添換上來,杯盤羅列,孟二舅吃至日西時分,告辭去了。

西門慶送了回來,還和伯爵吃了一回。只見買了兩座庫來,西門慶委付陳敬濟裝庫。問月娘尋出李瓶兒兩套錦衣,攪金銀錢紙裝在庫內。因向伯爵說:「今日是他六七,不念經,燒座庫兒。」伯爵道:「好快光陰,嫂子又早沒了個半月了。」西門慶道:「這出月初五日是他斷七,少不的替他念個經兒。」伯爵道:「這遭哥念佛經罷了。」西門慶道:「大房下說,他在時,因生小兒,許了些《血盆經懺》,許下家中走的兩個女僧做首座,請幾眾尼僧,替他禮拜幾捲懺兒罷了。」說畢,伯爵見天晚,說道:「我去罷。只怕你與嫂子燒紙。」又深深打恭說:「蒙哥厚情,死生難忘!」西門慶道:「難忘不難忘,我兒,你休推夢裡睡哩!你眾娘到滿月那日,買禮都要去哩。」伯爵道:「又買禮做甚?我就頭著地,好歹請眾嫂子到寒家光降光降。」西門慶道:「到那日,好歹把春花兒那奴才收拾起來,牽了來我瞧瞧。」伯爵道:「你春姨他說來,有了兒子,不用著你了。」西門慶道:「不要慌,我見了那奴才和他答話。」伯爵笑的去了。

西門慶令小廝收了傢伙,走到李瓶兒房裡。陳敬濟和玳安已把庫裝封停當。那日玉皇廟、永福寺、報恩寺都送疏來。西門慶看著迎春擺設羹飯完備,下出匾食來,點上香燭,使繡春請了吳月娘眾人來。西門慶與李瓶兒燒了紙,抬出庫去,教敬濟看著,大門首焚化。正是:
\begin{quote}
芳魂料不隨灰死,再結來生未了緣。
\end{quote}
