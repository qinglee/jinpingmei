
\chapter{韓畫士傳真作遺愛 西門慶觀戲動深悲}

詩曰:
\begin{quote}
香杳美人違,遙遙有所思。
幽明千里隔,風月兩邊時。
相對春那劇,相望景偏遲。
當由分別久,夢來還自疑。
\end{quote}

話說西門慶被應伯爵勸解了一回,拭淚令小廝後邊看飯去了。不一時,吳大舅、吳二舅都到了。靈前行禮畢,與西門慶作揖,道及煩惱之意。請至廂房中,與眾人同坐。

玳安走至後邊,向月娘說:「如何?我說娘每不信,怎的應二爹來了,一席話說的爹就吃飯了。」金蓮道:「你這賊,積年久慣的囚根子,鎮日在外邊替他做牽頭,有個拿不住他性兒的!」玳安道:「從小兒答應主子,不知心腹?」月娘問道:「那幾個陪他吃飯?」玳安道:「大舅、二舅才來,和溫師父,連應二爹、謝爹、韓伙計、姐夫,共爹八個人哩。」月娘道:「請你姐夫來後邊吃罷了,也擠在上頭!」玳安道:「姐夫坐下了。」月娘吩咐:「你和小廝往廚房裡拿飯去。你另拿甌兒粥與他吃,怕清早晨不吃飯。」玳安道:「再有誰?止我在家,都使出報喪、買東西,王經,又使他往張親家爹那裡借雲板去了。」月娘道:「書童那奴才和你拿去是的,怕打了他紗帽展翅兒!」玳安道:「書童和畫童兩個在靈前,一個打磐,一個伺候焚香燒紙哩。春鴻,爹又使他跟賁四換絹去了——嫌絹不好,要換六錢一匹的破孝。」月娘道:「論起來,五錢的也罷,又巴巴兒換去!」又道:「你叫下畫童兒那小奴才,和他快拿去,只顧還挨甚麼!」玳安於是和畫童兩個,大盤大碗拿到前邊,安放八仙桌席。眾人正吃著飯,只見平安拿進手本來稟:「夏老爹差寫字的,送了三班軍衛來這裡答應。」西門慶看了,吩咐:「討三錢銀子賞他。寫期服生帖兒回你夏老爹:多謝了!」

一面吃畢飯,收了傢伙。只見來保請的畫師韓先生來到。西門慶與他行畢禮,說道:「煩先生揭白傳個神子兒。」那韓先生道:「小人理會得。」吳大舅道:「動手遲了些,只怕面容改了。」韓先生道:「也不妨,就是揭白也傳得。」正吃茶畢,忽見平安來報:「門外花大舅來了。」西門慶陪花子由靈前哭涕了一回,見畢禮數,與眾人一處,因問:「甚麼時侯?」西門慶道:「正醜時斷氣。臨死還伶伶俐俐說話兒,剛睡下,丫頭起來瞧,就沒了氣兒。」因見韓先生旁邊小童拿著屏插,袖中取出描筆顏色來,花子由道:「姐夫如今要傳個神子?」西門慶道:「我心裡疼他,少不得留個影像兒,早晚看著,題念他題念兒。」一面吩咐後邊堂客躲開,掀起帳子,領韓先生和花大舅眾人到跟前。這韓先生揭起千秋幡,打一觀看,見李瓶兒勒著鴉青手帕,雖故久病,其顏色如生,姿容不改,黃懨懨的,嘴唇兒紅潤可愛。那西門慶由不的掩淚而哭。來保與琴童在旁捧著屏插、顏色。韓先生一見就知道了。眾人圍著他求畫,應伯爵便道:「先生,此是病容,平昔好時,還生的面容飽滿,姿容秀麗。」韓先生道:「不須尊長吩咐,小人知道。敢問老爹:此位老夫人,前者五月初一日曾在岳廟裡燒香,親見一面,可是否?」西門慶道:「正是。那時還好哩。先生,你用心想著,傳畫一軸大影、一軸半身,靈前供養,我送先生一匹緞子、十兩銀子。」韓先生道:「老爹吩咐,小人無不用心。」須臾,描染出個半身來,端的玉貌幽花秀麗,肌膚嫩玉生香。拿與眾人瞧,就是一幅美人圖兒。西門慶看了,吩咐玳安:「拿與你娘每瞧瞧去,看好不好。有那些兒不是,說來好改。」

玳安拿到後邊,向月娘道:「爹說叫娘每瞧瞧,六娘這影畫得如何,那些兒不象,說出去教韓先生好改。」月娘道:「成精鼓搗,人也不知死到那裡去了,又描起影來了。」潘金蓮接說道:「那個是他的兒女?畫下影,傳下神,好替他磕頭禮拜!到明日六個老婆死了,畫六個影才好。」孟玉樓和李嬌兒接過來觀看,說道:「大娘,你來看,李大姐這影,倒象好時模樣,打扮的鮮鮮的,只是嘴唇略扁了些。」月娘看了道:「這左邊額頭略低了些,他的眉角還彎些。虧這漢子,揭白怎的畫來!」玳安道:「他在廟上曾見過六娘一面,剛纔想著,就畫到這等模樣。」

少頃,只見王經進來說道:「娘每看了,就教拿出去。喬親家爹來了,等喬親家爹瞧哩。」玳安走到前邊,向韓先生道:「裡邊說來,嘴唇略扁了些,左額角稍低些,眉還要略放彎些兒。」韓先生道:「這個不打緊。」隨即取描筆改過了,呈與喬大戶瞧。喬大戶道:「親家母這幅尊像,真畫得好,只少了口氣兒。」西門慶滿心歡喜,一面遞了三鐘酒與韓先生,管待了酒飯,又教取出一匹尺頭、十兩白金與韓先生,教他:「先攢造出半身來,就要掛,大影,不誤出殯就是了。俱要用大青大綠,冠袍齊整,綾裱牙軸。」韓先生道:「不必吩咐,小人知道。」領了銀子,教小童拿著插屏,拜辭出門。喬大戶與眾人又看了一回做成的棺木,便道:「親家母今已小殮罷了?」西門慶道:「如今仵作行人來就小殮。大殮還等到三日。」喬大戶吃畢茶,就告辭去了。

不一時,仵作行人來伺候,紙札打捲,鋪下衣衾,西門慶要親與他開光明,強著陳敬濟做孝子,與他抿了目,西門慶旋尋出一顆胡珠,安放在他口裡。登時小殮停當,照前停放端正,合家大小哭了一場。來興又早冥衣鋪里,做了四座堆金瀝粉捧盆巾盥櫛毛女兒,一邊兩座擺下。靈前的彝爐商瓶、燭臺香盒,教錫匠打造停當,擺在桌上,耀日爭輝。又兌了十兩銀子,教銀匠打了三副銀爵盞。又與應伯爵定管喪禮簿籍:先兌了五百兩銀子、一百弔錢來,委付與韓伙計管帳;賁四與來興兒管買辦,兼管外廚房;應伯爵、謝希大、溫秀才、甘伙計輪番陪待弔客;崔本專管付孝帳;來保管外庫房;王經管酒房;春鴻與畫童專管靈前伺候;平安與四名排軍,單管人來打雲板、捧香紙;又叫一個寫字帶領四名排軍,在大門首記門簿,值念經日期,打傘挑幡幢。都派委已定,寫了告示,貼在影壁上,各遵守去訖。只見皇莊上薛內相差人送了六十根杉條、三十條毛竹、三百領蘆席、一百條麻繩,西門慶賞了來人五錢銀子,拿期服生回帖兒打發去了。吩咐搭採匠把棚起脊搭大些,留兩個門走,把影壁夾在中間,前廚房內還搭三間罩棚,大門首扎七間榜棚,請報恩寺十二眾僧人先念倒頭經,每日兩個茶酒伺候茶水。

花大舅、吳二舅坐了一回,起身去了。西門慶交溫秀才寫孝帖兒,要刊去,令寫「荊婦奄逝」,溫秀才悄悄拿與應伯爵看,伯爵道:「這個禮上說不通。見有如今吳家嫂子在正室,如何使得?這一齣去,不被人議論!就是吳大哥,心內也不自在。等我慢慢再與他講,你且休要寫著。」陪坐至晚,各散歸家去了。

西門慶晚夕也不進後邊去,就在李瓶兒靈旁裝一張涼床,拿圍屏圍著,獨自宿歇,止春鴻、書童兒近前伏侍。天明便往月娘房裡梳洗,穿戴了白唐巾孝冠孝衣、白絨襪、白履鞋,絰帶隨身。

第二日清晨,夏提刑就來探喪弔問,慰其節哀。西門慶還禮畢,溫秀才相陪,待茶而去。到門首,吩咐寫字的:「好生答應,查有不到的排軍,呈來衙門內懲治。」說畢,騎馬去了。西門慶令溫秀才發帖兒,差人請各親眷,三日誦經,早來吃齋。後晌,鋪排來收拾道場,懸掛佛像,不必細說。

那日,吳銀兒打聽得知,坐轎子來靈前哭泣上紙。到後邊,月娘相接。吳銀兒與月娘磕頭,哭道:「六娘沒了,我通一字不知,就沒個人兒和我說聲兒。可憐,傷感人也!」孟玉樓道:「你是他乾女兒,他不好了這些時,你就不來看他看兒?」吳銀兒道:「好三娘,我但知道,有個不來看的?說句假就死了!委實不知道。」月娘道:「你不來看你娘,他倒還掛牽著你,留下件東西兒,與你做一念兒,我替你收著哩。」因令小玉:「你取出來與銀姐看。」小玉走到裡面,取出包袱,打開是一套緞子衣服、兩根金頭簪兒、一枝金花。把吳銀兒哭的淚如雨點相似,說道:「我早知他老人家不好,也來伏侍兩日兒。」說畢,一面拜謝了月娘。月娘待茶與他吃,留他過了三日去。

到三日,和尚打起磐子,道場誦經,挑出紙錢去。合家大小都披麻帶孝。陳敬濟穿重孝絰巾,佛前拜禮,街坊鄰舍、親朋長官都來弔問,上紙祭奠者,不論其數。陰陽徐先生早來伺候大殮。祭告已畢,抬屍入棺,西門慶交吳月娘又尋出他四套上色衣服來,裝在棺內,四角又安放了四錠小銀子兒。花子由說:「姐夫,倒不消安他在裡面,金銀日久定要出世,倒非久遠之計。」西門慶不肯,定要安放。不一時,放下了七星板,擱上紫蓋,仵作四面用長命釘一齊釘起來,一家大小放聲號哭。西門慶亦哭的呆了,口口聲聲只叫:「我的年小的姐姐,再不得見你了!」良久哭畢,管待徐先生齋饌,打發去了。闔家伙計都是巾帶孝服,行香之時,門首一片皆白。溫秀才舉薦,北邊杜中書來題銘旌。杜中書名子春,號雲野,原侍真宗寧和殿,今坐閑在家,西門慶備金帛請來。在捲棚內備果盒,西門慶親遞三杯酒,應伯爵與溫秀才相陪。鋪大紅官紵題旌,西門慶要寫「詔封錦衣西門恭人李氏柩」十一字,伯爵再三不肯,說:「見有正室夫人在,如何使得!」杜中書道:「曾生過子,於禮也無礙。」講了半日,去了「恭」字,改了「室人」。溫秀才道:「恭人系命婦,有爵;室人乃室內之人,只是個渾然通常之稱。」於是用白粉題畢,「詔封」二字貼了金,懸於靈前。又題了神主。叩謝杜中書,管待酒饌,拜辭而去。

那日,喬大戶、吳大舅、花大舅、韓姨夫、沈姨夫各家都是三牲祭桌來燒紙。喬大戶娘子並吳大妗子、二妗子、花大妗子,坐轎子來弔喪,祭祀哭泣。月娘等皆孝髻,頭須系腰,麻布孝裙,出來回禮舉哀,讓後邊待茶擺齋。惟花大妗子與花大舅便是重孝直身,餘者都是輕孝。那日李桂姐打聽得知,坐轎子也來上紙,看見吳銀兒在這裡,說道:「你幾時來的?怎的也不會我會兒?好人兒,原來只顧你!」吳銀兒道:「我也不知道娘沒了,早知也來看看了。」月娘後邊管待,俱不必細說。

須臾過了,看看到首七,又是報恩寺十六眾上僧,朗僧官為首座,引領做水陸道場,誦《法華經》,拜三昧水懺。親朋伙計無不畢集。那日,玉皇廟吳道官來上紙弔孝,就攬二七經,西門慶留在捲棚內吃齋。忽見小廝來報:「韓先生送半身影來。」眾人觀看,但見頭戴金翠圍冠,雙鳳珠子挑牌、大紅妝花袍兒,白馥馥臉兒,儼然如生。西門慶見了,滿心歡喜。懸掛材頭,眾人無不誇獎:「只少口氣兒!」一面讓捲棚內吃齋,囑咐:「大影還要加工夫些。」韓先生道:「小人隨筆潤色,豈敢粗心!」西門慶厚賞而去。

午間,喬大戶來上祭,豬羊祭品、金銀山、緞帛彩繒、冥紙炷香共約五十餘抬,地弔高撬,鑼鼓細樂吹打,纓絡喧闐而至。西門慶與陳敬濟穿孝衣在靈前還禮。喬大戶邀了尚舉人、朱堂官、吳大舅、劉學官、花千戶、段親家七八位親朋,各在靈前上香。三獻已畢,俱跪聽陰陽生讀祝文曰:
\begin{quote}
維政和七年,歲次丁酉,九月庚申朔,越二十二日辛巳,眷生喬洪等謹以剛鬣柔毛庶羞之奠,致祭於故親家母西門孺人李氏之靈曰:嗚呼!孺人之性,寬裕溫良,治家勤儉,御眾慈祥,克全婦道,譽動鄉邦。閨閫之秀,蘭蕙之芳,夙配君子,效聘鸞凰。藍玉已種,浦珠已光。正期諧琴瑟於有永,享彌壽於無疆。胡為一病,夢斷黃粱?善人之歿,孰不哀傷?弱女襁褓,沐愛姻嬙。不期中道,天不從願,鴛伴失行。恨隔幽冥,莫睹行藏。悠悠情誼,寓此一觴。靈其有知,來格來歆。尚饗。
\end{quote}

官客祭畢,回禮畢,讓捲棚內桌席管待。然後喬大戶娘子、崔親家母、朱堂官娘子、尚舉人娘子、段大姐眾堂客女眷祭奠,地弔鑼鼓,靈前弔鬼判隊舞。吳月娘陪著哭畢,請去後邊待茶設席,三湯五割,俱不必細說。

西門慶正在捲棚內陪人吃酒,忽前邊打的雲板響。答應的慌慌張張進來稟報:「本府胡爺上紙來了,在門首下轎子。」慌的西門慶連忙穿孝衣,靈前伺候。即使溫秀才衣巾素服出迎,左右先捧進香紙,然後胡府尹素服金帶進來。許多官吏圍隨,扶衣搊帶,到了靈前,春鴻跪著,捧的香高高的,上了香,展拜兩禮。西門慶便道:「老先生請起,多有勞動。」連忙下來回禮。胡府尹道,「令夫人幾時沒了?學生昨日才知。弔遲,弔遲!」西門慶道:「側室一疾不救,辱承老先生枉弔。」溫秀才在旁作揖畢,請到廳上待茶一杯,胡府尹起身,溫秀才送出大門,上轎而去。上祭人吃至後晌方散。

第二日,院中鄭愛月兒家來上紙。愛月兒進至靈前,燒了紙。月娘見他抬了八盤餅饊、三牲湯飯來祭奠,連忙討了一匹整絹孝裙與他。吳銀兒與李桂姐都是三錢奠儀,告西門慶說。西門慶道:「值甚麼,每人都與他一匹整絹就是了。」月娘邀到後邊房裡,擺茶管待,過夜。

晚夕,親朋伙計來伴宿,叫了一起海鹽子弟搬演戲文。李銘、吳惠、鄭奉、鄭春都在這裡答應。西門慶在大棚內放十五張桌席,為首的就是喬大戶、吳大舅、吳二舅、花大舅、沈姨夫、韓姨夫、倪秀才、溫秀才、任醫官、李智、黃四、應伯爵、謝希大、祝實念、孫寡嘴、白賚光、常峙節、傅日新、韓道國、甘出身、賁第傳、吳舜臣、兩個外甥,還有街坊六七位人,都是開桌兒。點起十數枝大燭來,堂客便在靈前圍著圍屏,垂簾放桌席,往外觀戲。當時眾人祭奠畢,西門慶與敬濟回畢禮,安席上坐。下邊戲子打動鑼鼓,搬演的是韋皋、玉簫女兩世姻緣《玉環記》。不一時弔場,生扮韋皋,唱了一回下去。貼旦扮玉簫,又唱了一回下去。廚役上湯飯割鵝。應伯爵便向西門慶說:「我聞的院里姐兒三個在這裡,何不請出來,與喬老親家、老舅席上遞杯酒兒。他倒是會看戲文,倒便益了他!」西門慶便使玳安進入說去:「請他姐兒三個出來。」喬大戶道:「這個卻不當。他來弔喪,如何叫他遞起酒來?」伯爵道:「老親家,你不知,象這樣小淫婦兒,別要閑著他。——快與我牽出來!你說應二爹說,六娘沒了,只當行孝順,也該與俺每人遞杯酒兒。」玳安進去半日,說:「聽見應二爹在坐,都不出來哩。」伯爵道:「既恁說,我去罷。」走了兩步,又回坐下。西門慶笑道:「你怎的又回了?」伯爵道:「我有心待要扯那三個小淫婦出來,等我罵兩句,出了我氣,我才去。」落後又使玳安請了一遍,三個才慢條條出來。都一色穿著白綾對衿襖兒、藍緞裙子,向席上不端不正拜了拜兒,笑嘻嘻立在旁邊。應伯爵道:「俺每在這裡,你如何只顧推三阻四,不肯出來?」那三個也不答應,向上邊遞了回酒,設一席坐著。下邊鼓樂響動,關目上來,生扮韋皋,凈扮包知木,同到勾欄里玉簫家來。那媽兒出來迎接,包知木道:「你去叫那姐兒出來。」媽云:「包官人,你好不著人,俺女兒等閑不便出來。說不得一個『請』字兒,你如何說『叫他出來』?」那李桂姐向席上笑道:「這個姓包的,就和應花子一般,就是個不知趣的蹇味兒!」伯爵道:「小淫婦,我不知趣,你家媽怎喜歡我?」桂姐道:「他喜歡你?過一邊兒!」西門慶道:「看戲罷,且說甚麼。再言語,罰一大杯酒!」那伯爵才不言語了。那戲子又做了一回,並下。

廳內左邊弔帘子看戲的,是吳大妗子、二妗子、楊姑娘、潘姥姥、吳大姨、孟大姨、吳舜臣媳婦鄭三姐、段大姐,並本家月娘姊妹;右邊弔帘子看戲的,是春梅、玉簫、蘭香、迎春、小玉,都擠著觀看。那打茶的鄭紀,正拿著一盤果仁泡茶從簾下過,被春梅叫住,問道:「拿茶與誰吃?」鄭紀道:「那邊六妗子娘每要吃。」這春梅取一盞在手。不想小玉聽見下邊扮戲的旦兒名字也叫玉簫,便把玉簫拉著說道:「淫婦,你的孤老漢子來了。鴇子叫你接客哩,你還不出去。」使力往外一推,直推出帘子外,春梅手裡拿著茶,推潑一身。罵玉簫:「怪淫婦,不知甚麼張致,都頑的這等!把人的茶都推潑了,早是沒曾打碎盞兒。」西門慶聽得,使下來安兒來問:「誰在裡面喧嚷?」春梅坐在椅上道:「你去就說,玉簫浪淫婦,見了漢子這等浪。」那西門慶問了一回,亂著席上遞酒,就罷了。月娘便走過那邊數落小玉:「你出來這一日,也往屋裡瞧瞧去。都在這裡,屋裡有誰?」小玉道:「大姐剛纔後邊去的,兩位師父也在屋裡坐著。」月娘道:「教你們賊狗胎在這裡看看,就恁惹是招非的。」春梅見月娘過來,連忙立起身來說道:「娘,你問他。都一個個只象有風病的,狂的通沒些成色兒,嘻嘻哈哈,也不顧人看見。」那月娘數落了一回,仍過那邊去了。

那時,喬大戶與倪秀才先起身去了。沈姨夫與任醫官、韓姨夫也要起身,被應伯爵攔住道:「東家,你也說聲兒。俺每倒是朋友,不敢散,一個親家都要去。沈姨夫又不隔門,韓姨夫與任大人、花大舅都在門外。這咱晚三更天氣,門也還未開,慌的甚麼?都來大坐回兒,左右關目還未了哩。」西門慶又令小廝提四壇麻姑酒,放在面前,說:「列位只了此四壇酒,我也不留了。」因拿大賞鐘放在吳大舅面前,說道:「那位離席破坐說起身者,任大舅舉罰。」於是眾人又復坐下了。西門慶令書童:「催促子弟,快弔關目上來,吩咐揀著熱鬧處唱罷。」須臾打動鼓板,扮末的上來,請問西門慶:「『寄真容』那一折可要唱?」西門慶道:「我不管你,只要熱鬧。」貼旦扮玉簫唱了回。西門慶看唱到「今生難會面,因此上寄丹青」一句,忽想起李瓶兒病時模樣,不覺心中感觸起來,止不住眼中淚落,袖中不住取汗巾兒搽拭。又早被潘金蓮在簾內冷眼看見,指與月娘瞧,說道:「大娘,你看他好個沒來頭的行貨子,如何吃著酒,看見扮戲的哭起來?」盂玉樓道:「你聰明一場,這些兒就不知道了?樂有悲歡離合,想必看見那一段兒觸著他心,他睹物思人,見鞍思馬,才掉淚來。」金蓮道:「我不信。打談的掉眼淚——替古人耽憂,這些都是虛。他若唱的我淚出來,我才算他好戲子。」月娘道:「六姐,悄悄兒,咱每聽罷。」玉樓因向大妗子道:「俺六姐不知怎的,只好快說嘴。」

那戲子又做了一回,約有五更時分,眾人齊起身。西門慶拿大杯攔門遞酒,款留不住,俱送出門。看收了傢伙,留下戲廂:「明日有劉公公、薛公公來祭奠,還做一日。」眾戲子答應。管待了酒飯,歸下處歇去了。李銘等四個亦歸家不題。西門慶見天色已將曉,就歸後邊歇息去了。正是,得多少——
\begin{quote}
紅日映窗寒色淺,淡煙籠竹曙光微。
\end{quote}
