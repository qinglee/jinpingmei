
\chapter{假弟妹暗續鸞膠 真夫婦明諧花燭}

詞曰:
\begin{quote}
追悔當初辜深願,經年價,兩成幽怨。任越水吳山,似屏如障堪遊玩,奈獨自慵抬眼。
賞煙花,聽弦管,徒歡娛,轉加腸斷。總時轉丹青,強拈書信頻頻看,又曾似親眼見。
\end{quote}

話說陳敬濟,到於守備府中,下了馬,張勝先進去稟報春梅。春梅分付,教他在外邊班直房內,用香湯沐浴了身體,後邊使養娘包出一套新衣服靴帽來,與他更換了。然後稟了春梅。那時守備還未退廳,春梅請敬濟到後堂,盛妝打扮,出來相見。這敬濟進門就望春梅拜了四雙八拜,讓姐姐受禮。那春梅受了半禮,對面坐下。敘了寒溫離別之情,彼此皆眼中垂淚。春梅恐怕守備退廳進來,見無人在根前,使眼色與敬濟,悄悄說:「等住回他若問你,只說是姑表兄弟。我大你一歲,二十五歲了,四月廿五日午時生的。」敬濟道:「我知道了。」不一時,丫鬟拿上茶來,兩人吃了茶,春梅便問:「你一向怎麼出了家做了道士?守備不知是我的親,錯打了你,悔的要不的。若不是那時就留下你,爭奈有雪娥那賤人在這裡,不好安插你的。所以放你去了。落後打發了那賤人,才使張勝到處尋你不著,誰知你在城外做工,流落至此地位。」敬濟道:「不瞞姐姐說,一言難盡。自從與你相別,要娶六姐,我父親死在東京,來遲了,不曾娶成,被武鬆殺了。聞得你好心,葬埋了他永福寺,我也到那裡燒紙來。落後又把俺娘沒了,剛打發喪事出去,被人坑陷了資本。來家又是大姐死了,被俺丈母那淫婦告了一狀,床帳妝奩,都搬的去了。打了一場官司,將房兒賣了,弄的我一貧如洗。多虧了俺爹朋友王杏庵周濟,把我才送到臨清晏公廟那裡出家。不料又被光棍打了,拴到咱府中。自從咱府中出去,投親不理,投友不顧,因此在寺內傭工。多虧姐姐掛心,使張管家尋將我來,得見姐姐一面,猶如再世為人了。」說到傷心處,兩個都哭了。

正說話中間,只見守備退廳,左右掀開帘子,守備進來。這陳敬濟向前,倒身下拜。慌的守備答禮相還,說:「嚮日不知是賢弟,被下人隱瞞,誤有衝撞,賢弟休怪。」敬濟道:「不才有話,一向缺禮,有失親近,望乞恕罪。」又磕下頭去。守備一手扯起,讓他上坐。敬濟乖覺,那裡肯,務要拉下椅兒旁邊坐了。守備關席,春梅陪他對坐下。須臾,換茶上來。吃畢,守備便問:「賢弟貴庚?一向怎的不見?如何出家?」敬濟使告說:「小弟虛度二十四歲。俺姐姐長我一歲,是四月二十五日午時生。向因父母雙亡,家業凋喪,妻又沒了,出家在晏公廟。不知家姐嫁在府中,有失探望。」守備道:「自從賢弟那日去後,你令姐晝夜憂心,常時啾啾唧唧,不安直到如今。一向使人找尋賢弟不著,不期今日相會,實乃三生有緣。」

看官聽說,若論周守備與西門慶相交,也該認得陳敬濟,原來守備為人老成正氣,舊時雖然來往,並不留心管他家閑事。就是時常宴會,皆同的是荊都監、夏提刑一班官長,並未與敬濟見面。況前日又做了道士一番,那裡還想的到西門慶家女婿?所以被他二人瞞過,只認是春梅姑表兄弟。一面分付左右放桌兒,安排酒上來。須臾,擺設許多杯盤餚饌,湯飯點心,堆滿桌上,銀壺玉盞,酒泛金波。守備相陪敘話,吃至晚來,掌上燈燭方罷。守備分付家人周仁,打掃西書院乾凈,那裡床帳都有。春梅拿出兩床鋪蓋衾枕,與他安歇。又撥了一個小廝喜兒答應他。又包出兩套綢絹衣服來,與他更換。每日飯食,春梅請進後邊吃。正是:
\begin{quote}
一朝時運至,半點不由人。
\end{quote}
光陰迅速,日月如梭,但見:
\begin{quote}
行見梅花臘底,忽逢元旦新正。
不覺艷杏盈枝,又早新荷貼水。
\end{quote}

敬濟在守備府里,住了個月有餘。一日是四月二十五日,春梅的生日。吳月娘那邊買了禮來,一盤壽桃,一盤壽麵,兩隻湯鵝,四隻鮮雞,兩盤果品,一壇南酒。玳安穿青衣拿貼兒送來。守備正在廳上坐的,門上人稟報,抬進禮來。玳安遞上貼兒,扒在地下磕頭。守備看了禮貼兒,說道:「多承你奶奶費心,又送禮來。」一面分付家人:「收進禮去,討茶來與大官兒吃。把禮貼教小伴當送與你舅收了。封了一方手帕、三錢銀子與大官兒,抬盒人錢一百文,拿回貼兒,多上覆。」說畢,守備穿了衣服,就起身拜人去了。玳安只顧在廳前伺候,討回貼兒。只見一個年少的,戴著瓦楞帽兒,穿著青紗道袍,涼鞋凈襪,從角門裡走出來,手中拿著貼兒賞錢,遞與小伴當,一直往後邊去了。「可霎作怪,模樣倒好相陳姐夫一般。他如何卻在這裡?」只見小伴當遞與玳安手帕銀錢,打發出門。

到於家中,回月娘話。見回貼上寫著「周門龐氏斂衽拜」。月娘便問:「你沒見你姐?」玳安道:「姐姐倒沒見,倒見姐夫來。」月娘笑道:「怪囚,你家倒有恁大姐夫!守備好大年紀,你也叫他姐夫。」玳安道:「不是守備,是咱家的陳姐夫。我初進去,周爺正在廳上,我遞上貼兒與他磕了頭,他說:『又生受你奶奶送重禮來。』分付伴當拿茶與我吃,『把貼兒拿與你舅收了,討一方手帕、三錢銀子與大官兒,抬盒人是一百文錢。』說畢,周爺穿衣服出來,上馬拜人去了。半日,只見他打角門裡出來,遞與伴當回貼賞賜,他就進後邊去了,我就押著盒擔出來。不是他卻是誰?」月娘道:「怪小囚兒,休胡說白道的。那羔子知道流落在那裡討吃?不是凍死,就是餓死,他平白在那府里做甚麼?守備認的他甚麼毛片兒,肯招攬下他?」玳安道:「奶奶敢和我兩個賭,我看得千真萬真,就燒的成灰骨兒我也認的。」月娘道:「他穿著甚麼?」玳安道:「他戴著新瓦楞帽兒,金簪子。身穿著青紗道袍,涼鞋凈襪。吃的好了。」月娘道:「我不信,不信。」這裡說話不題。

卻說陳敬濟進入後邊,春梅還在房中鏡臺前搽臉,描畫雙蛾。敬濟拿吳月娘禮貼兒與他看。因問:「他家如何送禮來與你?是那裡緣故?」這春梅便把清明郊外,永福寺撞遇月娘相見的話,訴說一遍。後來怎生平安兒偷瞭解當鋪頭面,吳巡簡怎生夾打平安兒,追問月娘姦情之事,薛嫂又怎生說人情,守備替他處斷了事,落後他家買禮來相謝。正月里,我往他家與孝哥兒做生日,勾搭連環到如今。他許下我生日買禮來看我一節,說了一遍。敬濟聽了,把眼瞅了春梅一眼,說:「姐姐,你好沒志氣。想著這賊淫婦那咱,把咱姐兒們生生的拆散開了,又把六姐命喪了,永世千年,門裡門外不相逢才好,反替他去說人情兒。那怕那吳典恩拷打玳安小廝,供出姦情來,隨他那淫婦一條繩子拴去,出醜見官,管咱每大腿事?他沒和玳安小廝有姦,怎的把丫頭小玉配與他?有我早在這裡,我斷不教你替他說人情。他是你我仇人,又和他上門往來做甚麼?六月連陰——想他好情兒!」幾句話,說得春梅閉口無言。這春梅道:「過往勾當,也罷了,還是我心好,不念舊仇。」敬濟道:「如今人好心不得這報哩。」春梅道:「他既送了禮,莫不白受他的?他還等著我這裡人請他去哩。」敬濟道:「今後不消理那淫婦了,又請他怎的?」春梅道:「不請他又不好意思的。丟個貼兒與他,來不來隨他就是了。他若來時,你在那邊書院內,休出來見他,往後咱不招惹他就是了。」敬濟惱的一聲兒不言語,走到前邊,寫了貼兒。春梅使家人周義去請吳月娘。月娘打扮出門,教奶子如意兒抱著孝哥兒,坐著一頂小轎,玳安跟隨,來到府中。春梅、孫二娘都打扮出來,迎接至後廳相見,敘禮坐下。如意兒抱著孝哥兒,相見磕頭畢。敬濟躲在那邊書院內,不走出來,由著春梅、孫二娘在後廳擺茶安席遞酒。叫了兩個妓女韓玉釧、鄭嬌兒彈唱,俱不必細說。

玳安在前邊廂房內管待。只見一個小伴當,打後邊拿著一盤湯飯點心下飯,往西角門書院中走。玳安便問他拿與誰吃,小伴當說:「是與舅吃的。」玳安道:「代舅姓甚麼?」小伴當道:「姓陳。」這玳安賊,悄悄後邊跟著他到西書院。小伴當便掀帘子進去,放卓兒吃。這玳安悄悄走出外來,依舊坐在廂房內。直待天晚,家中燈籠來接,吳月娘轎子起身。到家,一五一十告訴月娘說:「果然陳姐夫在他家居住。」自從春梅這邊被敬濟把攔,兩家都不相往還。正是:
\begin{quote}
誰知豎子多間阻,一念翻成怨恨媒。
\end{quote}

敬濟在府中與春梅暗地勾搭,人都不知。或守備不在,春梅就和敬濟在房中吃飯吃酒,閑時下棋調笑,無所不至。守備在家,便使丫頭小廝拿飯往書院與他吃。或白日里,春梅也常往書院內,和他坐半日,方歸後邊來。彼此情熱,俱不必細說。

一日,守備領人馬出巡,正值五月端午佳節。春梅在西書院花亭上置了一卓酒席,和孫二娘、陳敬濟吃雄黃酒,解粽歡娛。丫鬟侍妾都兩邊侍奉。春梅令海棠、月桂兩個侍妾在席前彈唱。當下直吃到炎光西墜、微雨生涼的時分。春梅拿起大金荷花杯來相勸。酒過數巡,孫二娘不勝酒力,起身先往後邊房中看去了。獨落下春梅和敬濟在花亭上吃酒,猜枚行令,你一杯,我一杯。不一時,丫鬟掌上紗燈來,養娘金匱、玉堂打發金哥兒睡去了。敬濟輸了,便走入書房內躲酒不出來。這春梅先使海棠來請,見敬濟不去,又使月桂來,分付:「他不來,你好歹與我拉將來。拉不將來,回來把你這賤人打十個嘴巴。」這月桂走至西書房中,推開門,見敬濟歪在床上,推打鼾睡,不動。月桂說:「奶奶叫我來請你老人家,請不去,要打我哩。」那敬濟口裡喃喃吶吶說:「打你不乾我事。我醉了,吃不的了。」被月桂用手拉將起來,推著他:「我好歹拉你去,拉不將你去,也不算好漢。」推拉的敬濟急了,黑影子里佯裝著醉,作耍當真,摟了月桂在懷裡就親個嘴。那月桂亦發上頭上腦說:「人好意叫你,你就大不正,倒做這個營生。」敬濟道:「我的兒,你若肯了,那個好意做大不成?」又按著親了個嘴,方走到花亭上。月桂道:「奶奶要打我,還是我把舅拉將來了。」春梅令海棠斟上大鐘,兩個下盤棋,賭酒為樂。當下你一盤,我一盤,熬的丫鬟都打睡去了。春梅又使月桂、海棠後邊取茶去,兩個在花亭上,解佩露相如之玉,朱唇點漢署之香。正是:得多少——
\begin{quote}
花陰曲檻燈斜照,旁有墜釵雙鳳翹。
\end{quote}
有詩為證:
\begin{quote}
花亭歡洽鬢雲斜,粉汗凝香沁絳紗。
深院日長人不到,試看黃鳥啄名花。
\end{quote}

兩個正幹得好,忽然丫鬟海棠送茶來:「請奶奶後邊去,金哥睡醒了,哭著尋奶奶哩。」春梅陪敬濟又吃了兩鐘酒,用茶嗽了口,然後抽身往後邊來。丫鬟收拾了家活,喜兒扶敬濟歸書房寢歇,不在話下。

一日,朝廷敕旨下來,命守備領本部人馬,會同濟州府知府張叔夜,徵剿梁山泊賊王宋江,早晚起身。守備對春梅說:「你在家看好哥兒,叫媒人替你兄弟尋上一門親事。我帶他個名字在軍門,若早僥幸得功,朝廷恩典,升他一官半職,於你面上,也有光輝。」這春梅應諾了。遲了兩三日,守備打點行裝,整率人馬,留下張勝、李安看家,止帶家人周仁跟了去。不題。

一日,春梅叫將薛嫂兒來,如此這般和他說:「他爺臨去分付,叫你替我兄弟尋門親事,你須尋個門當戶對好女兒,不拘十六七歲的也罷,只要好模樣兒,聰明伶俐些的。他性兒也有些厥劣。」薛嫂兒道:「我不知道他也怎的?不消你老人家分付。想著大姐那等的還嫌哩。」春梅道:「若是尋的不好,看我打你耳刮子不打?我要趕著他叫小妗子兒哩,休要當耍子兒。」說畢,春梅令丫鬟擺茶與他吃。只見陳敬濟進來吃飯。薛嫂向他道了萬福,說:「姑夫,你老人家一向不見,在那裡來?且喜呀,剛剛奶奶分付,交我替你老人家尋個好娘子,你怎麼謝我?」那陳敬濟把臉兒迸著不言語。薛嫂道:「老花子怎的不言語?」春梅道:「你休要叫他姑夫,那個已是揭過去的帳了,你只叫他陳舅就是了。」薛嫂道:「真該打,我這片子狗嘴,只要叫錯了,往後趕著你只叫舅爺罷。」那敬濟忍不住,撲吃的笑了,說道:「這個才可到我心上。」那薛嫂撒風撒痴,趕著打了他一下,說道:「你看老花子說的好話兒,我又不是你影射的,怎麼可在你心上?」連春梅也笑了。

不一時,月桂安排茶食與薛嫂吃了,說道:「我替你老人家用心踏著,有人家相應好女子兒,就來說。」春梅道:「財禮羹果,花紅酒禮,頭面衣服,不少他的,只要好人家好女孩兒,方可進入我門來。」薛嫂道:「我曉得,管情應的你老人家心便了。」良久,敬濟吃了飯,往前邊去了。薛嫂兒還坐著,問春梅:「他老人家幾時來的?」春梅便把出家做道士一節說了:「我尋得他來,做我個親人兒。」薛嫂道:「好好,你老人家有後眼。」又道:「前日你老人家好日子,說那頭他大娘來做生日來?」春梅道:「他先送禮來,我才使人請他,坐了一日去了。」薛嫂道:「我那日在一個人家鋪床,整亂了一日。心內要來,急的我要不的。」又問:「他陳舅,也見他那頭大娘來?」春梅道:「他肯下氣見他?為請他,好不和我亂成一塊。嗔我替他家說人情,說我沒志氣。那怕吳典恩打著小廝,攀扯他出官才好,管你腿事?你替他尋分上,想著他昔日好情兒?」薛嫂道:「他老人家也說的是,及到其間,也不計舊仇罷了。」春梅道:「咱既受了他禮,不請他來坐坐兒,又使不的。寧可教他不仁,休要咱不義。」薛嫂道:「怪不的你老人家有恁大福,休的心忒好了!」當下薛嫂兒說了半日話,提著花箱兒,拜辭出門。

過了兩日,先來說:「城裡朱千戶家小姐,今年十五歲,也好陪嫁,只是沒了娘的兒了。」春梅嫌小不要。又說應伯爵第二個女兒,年二十二歲。春梅又嫌應伯爵死了,在大爺手內聘嫁,沒甚陪送,也不成。都回出婚帖兒來。又遲了幾日,薛嫂兒送花兒來,袖中取出個婚貼兒,大紅段子上寫著:「開段鋪葛員外家大女兒,年二十歲,屬雞的,十一月十五日子時生,小字翠屏。」「生的上畫兒般模樣兒,五短身材,瓜子面皮,溫柔典雅,聰明伶俐,針指女工,自不必說。父母俱在,有萬貫錢財。在大街上開段子鋪,走蘇杭、南京,無比好人家。陪嫁都是南京床帳箱籠。」春梅道:「既是好,成了這家的罷。」就交薛嫂兒先通信去。那薛嫂兒連忙說去了。正是:
\begin{quote}
欲向繡房求艷質,須憑紅葉是良媒。
\end{quote}
有詩為證:
\begin{quote}
天仙機上系香羅,千里姻緣竟足多。
天上牛郎配織女,人間才子伴嬌娥。
\end{quote}

這裡薛嫂通了信來,葛員外家知是守備府里,情願做親,又使一個張媒人同說媒。春梅這裡備了兩抬茶葉、糧餅、羹果,教孫二娘坐轎子,往葛員外家插定女兒。回來對春梅說:「果然好個女子,生的一表人才,如花似朵,人家又相當。」春梅這裡擇定吉日,納採行禮。十六盤羹果茶餅,兩盤頭面,二盤珠翠,四抬酒,兩牽羊,一頂鬒髻,全副金銀頭面簪環之類。兩件羅段袍兒,四季衣服。其餘綿花布絹,二十兩禮銀,不必細說。陰陽生擇在六月初八日,準娶過門。春梅先問薛嫂兒:「他家那裡有陪床使女沒有?」薛嫂兒道:「床帳妝奩都有,只沒有使女陪床。」春梅道:「咱這裡買一個十三四歲丫頭子,與他房裡使喚,掇桶子倒水方便些。」薛嫂道:「有,我明日帶一個來。」

到次日,果然領了一個丫頭,說:「是商人黃四家兒子房裡使的丫頭,今年才十三歲。黃四因用下官錢糧,和李三還有咱家出去的保官兒,都為錢糧捉拿在監里追贓,監了一年多,家產盡絕,房兒也賣了。李三先死,拿兒子李活監著。咱家保官兒那兒僧寶兒,如今流落在外,與人家跟馬哩。」春梅道:「是來保?」薛嫂道:「他如今不叫來保,改了名字叫湯保了。」春梅道:「這丫頭是黃四家丫頭,要多少銀子?」薛嫂道:「只要四兩半銀子。緊等著要交贓去。」春梅道:「甚麼四兩半,與他三兩五錢銀子留下罷。」一面就交了三兩五錢雪花官銀與他,寫了文書。改了名字,喚做金錢兒。

話休饒舌,又早到六月初八。春梅打扮珠翠鳳冠,穿通袖大紅袍兒,束金鑲碧玉帶。坐四人大轎,鼓樂燈籠,娶葛家女子,奠雁過門。陳敬濟騎大白馬,揀銀鞍轡,青衣軍牢喝道。頭戴儒巾,穿著青段圓領,腳下粉底皂靴,頭上簪著兩支金花。正是:
\begin{quote}
久旱逢甘雨,他鄉遇故知。
洞房花燭夜,金榜掛名時。
\end{quote}
一番拆洗一番新。到守備府中,新人轎子落下。頭蓋大紅銷金蓋袱,添妝含飯,抱著寶瓶進入大門。陰陽生引入畫堂,先參拜了堂,然後歸到洞房。春梅安他兩口兒坐帳,然後出來。陰陽生撒帳畢,打發喜錢出門,鼓手都散了。敬濟與這葛翠屏小姐坐了回帳,騎馬打燈籠,往岳丈家謝親。吃的大醉而歸。晚夕女貌郎才,未免燕爾新婚,交媾雲雨。正是:得多少——
\begin{quote}
春點杏桃紅綻蕊,風欺楊柳綠翻腰。
\end{quote}

當夜敬濟與這葛翠屏小姐倒且是合得著。兩個被底鴛鴦,帳中鸞鳳,如魚似水,合巹歡娛。三日完飯,春梅在府廳後堂張筵掛採,鼓樂笙歌,請親眷吃會親酒,俱不必細說。每日春梅吃飯,必請他兩口兒同在房中一處吃。彼此以姑妗稱之,同起同坐。丫頭養娘、家人媳婦,誰敢道個不字?原來春梅收拾西廂房三間,與他做房,裡面鋪著床帳,糊的雪洞般齊整,垂著簾幃。外邊西書院,是他書房。裡面亦有床榻、幾席、古書並守備往來書柬拜貼,並各處遞來手本揭貼,都打他手裡過。春梅不時出來書院中,和他閑坐說話,兩個暗地交情。正是:
\begin{quote}
朝陪金谷宴,暮伴綺樓娃。
休道歡娛處,流光逐落霞。
\end{quote}
