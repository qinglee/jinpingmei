
\chapter{盼情郎佳人占鬼卦 燒夫靈和尚聽淫聲}

詞曰:
\begin{quote}
紅曙捲窗紗,睡起半拖羅袂。何似等閑睡起,到日高還未。
催花陣陣玉樓風,樓上人難睡。有了人兒一個,在眼前心裡。
\end{quote}

話說西門慶自娶了玉樓在家,燕爾新婚,如膠似漆。又遇陳宅使文嫂兒來通信,六月十二日就要娶大姐過門。西門慶促忙促急攢造不出床來,就把孟玉樓陪來的一張南京描金彩漆拔步床陪了大姐。三朝九日,足亂了一個多月,不曾往潘金蓮家去。把那婦人每日門兒倚遍,眼兒望穿。使王婆往他門首去尋,門首小廝知道是潘金蓮使來的,多不理他。婦人盼的緊,見婆子回了,又叫小女兒街上去尋。那小妮子怎敢入他深宅大院?只在門首踅探,不見西門慶就回來了。來家被婦人噦罵在臉上,怪他沒用,便要叫他跪著。餓到晌午,又不與他飯吃。此時正值三伏天道,婦人害熱,吩咐迎兒熱下水,伺候要洗澡。又做了一籠裹餡肉角兒,等西門慶來吃。身上只著薄紗短衫,坐在小凳上,盼不見西門慶到來,罵了幾句負心賊。無情無緒,用纖手向腳上脫下兩隻紅繡鞋兒來,試打一個相思卦。正是:
\begin{quote}
逢人不敢高聲語,暗卜金錢問遠人。
\end{quote}
有《山坡羊》為證:
\begin{quote}
凌波羅襪,天然生下,紅雲染就相思卦。似藕生芽,如蓮卸花,怎生纏得些兒大!柳條兒比來剛半叉。他不念咱,咱何曾不念他!倚著門兒,私下簾兒,悄呀,空叫奴被兒里叫著他那名兒罵。你怎戀煙花,不來我家!奴眉兒淡淡教誰畫?何處綠楊拴系馬?他辜負咱,咱何曾辜負他!
\end{quote}

婦人打了一回相思卦,不覺睏倦,就歪在床上盹睡著了。約一個時辰醒來,心中正沒好氣。迎兒問:「熱了水,娘洗澡也不洗?」婦人就問:「角兒蒸熟了?拿來我看。」迎兒連忙拿到房中。婦人用纖手一數,原做下一扇籠三十個角兒,翻來覆去只數得二十九個,便問:「那一個往那裡去了?」迎兒道:「我並沒看見,只怕娘錯數了。」婦人道:「我親數了兩遍,三十個角兒,要等你爹來吃。你如何偷吃了一個?好嬌態淫婦奴才,你害饞癆饞痞,心裡要想這個角兒吃!你大碗小碗吃搗不下飯去,我做下孝順你來!」便不由分說,把這小妮子跣剝去身上衣服,拿馬鞭子打了二三十下,打的妮子殺豬般也似叫。問著他:「你不承認,我定打你百數!」打的妮子急了,說道:「娘休打,是我害餓的慌,偷吃了一個。」婦人道:「你偷了,如何賴我錯數?眼看著就是個牢頭禍根淫婦!有那亡八在時,輕學重告,今日往那裡去了?還在我跟前弄神弄鬼!我只把你這牢頭淫婦,打下你下截來!」打了一回,穿上小衣,放他起來,吩咐在旁打扇。打了一回扇,口中說道:「賊淫婦,你舒過臉來,等我掐你這皮臉兩下子。」那妮子真個舒著臉,被婦人尖指甲掐了兩道血口子,才饒了他。

良久,走到鏡臺前,從新妝點出來,門簾下站立。也是天假其便,只見玳安夾著氈包,騎著馬,打婦人門首經過。婦人叫住,問他往何處去來。那小廝說話乖覺,常跟西門慶在婦人家行走,婦人常與他些浸潤,以此滑熟。一面下馬來,說道:「俺爹使我送人情,往守備府里去來。」婦人叫進門來,問道:「你爹家中有甚事,如何一向不來傍個影兒?想必另續上了一個心甜的姊妹了。」玳安道:「俺爹再沒續上姊妹,只是這幾日家中事忙,不得脫身來看六姨。」婦人道:「就是家中有事,那裡丟我恁個半月,音信不送一個兒!只是不放在心兒上。」因問玳安:「有甚麼事?你對我說。」那小廝嘻嘻只是笑,不肯說。婦人見玳安笑得有因,愈丁緊問道:「端的有甚事?」玳安笑道:「只說有椿事兒罷了,六姨只顧吹毛求疵問怎的?」婦人道:「好小油嘴兒,你不對我說,我就惱你一生。」小廝道:「我對六姨說,六姨休對爹說是我說的。」婦人道:「我決不對他說。」玳安就如此這般,把家中娶孟玉樓之事,從頭至尾告訴了一遍。這婦人不聽便罷,聽了由不得珠淚兒順著香腮流將下來。玳安慌了,便道:「六姨,你原來這等量窄,我故此不對你說。」婦人倚定門兒,長嘆了一口氣,說道:「玳安,你不知道,我與他從前以往那樣恩情,今日如何一旦拋閃了。」止不住紛紛落下淚來。玳安道:「六姨,你何苦如此?家中俺娘也不管著他。」婦人便道:「玳安,你聽告訴:
\begin{quote}
喬才心邪,不來一月。奴繡鴛衾曠了三十夜。他俏心兒別,俺痴心兒呆,不合將人十分熱。常言道容易得來容易舍。興,過也;緣,分也。」
\end{quote}

說畢又哭。玳安道:「六姨,你休哭。俺爹怕不也只在這兩日,他生日待來也。你寫幾個字兒,等我替你捎去,與俺爹看了,必然就來。」婦人道:「是必累你,請的他來。到明日,我做雙好鞋與你穿。我這裡也要等他來,與他上壽哩。他若不來,都在你小油嘴身上。」說畢,令迎兒把桌上蒸下的角兒,裝了一碟,打發玳安兒吃茶。一面走入房中,取過一幅花箋,又輕拈玉管,款弄羊毛,須臾,寫了一首《寄生草》。詞曰:
\begin{quote}
將奴這知心話,付花箋寄與他。想當初結下青絲發,門兒倚遍簾兒下,受了些沒打弄的耽驚怕。你今果是負了奴心,不來還我香羅帕。
\end{quote}

寫就,疊成一個方勝兒,封停當,付與玳安收了,道:「好歹多上覆他。待他生日,千萬來走走。奴這裡專望。」那玳安吃了點心,婦人又與數十文錢。臨出門上馬,婦人道:「你到家見你爹,就說六姨好不罵你。他若不來,你就說六姨到明日坐轎子親自來哩。」玳安道:「六姨,自吃你賣粉團的撞見了敲板兒蠻子叫冤屈——麻飯胳膽的帳。」說畢,騎馬去了。

那婦人每日長等短等,如石沉大海。七月將盡,到了他生辰。這婦人挨一日似三秋,盼一夜如半夏,等得杳無音信。不覺銀牙暗咬,星眼流波。至晚,只得又叫王婆來,安排酒肉與他吃了,向頭上拔下一根金頭銀簪子與他,央往西門慶家去請他來。王婆道:「這早晚,茶前酒後,他定也不來。待老身明日侵早請他去罷。」婦人道:「乾娘,是必記心,休要忘了!」婆子道:「老身管著那一門兒,肯誤了勾當?」這婆子非錢而不行,得了這根簪子,吃得臉紅紅,歸家去了。且說婦人在房中,香薰鴛被,款剔銀燈,睡不著,短嘆長吁。正是:得多少——
\begin{quote}
琵琶夜久殷勤弄,寂寞空房不忍彈。
\end{quote}

於是獨自彈著琵琶,唱一個《綿搭絮》:
\begin{quote}
誰想你另有了裙釵,氣的奴似醉如痴,斜倚定幃屏故意兒猜,不明白。怎生丟開?傳書寄柬,你又不來。你若負了奴的恩情,人不為仇天降災。
\end{quote}

婦人一夜翻來覆去,不曾睡著。巴到天明,就使迎兒:「過間壁瞧王奶奶請你爹去了不曾?」迎兒去不多時,說:「王奶奶老早就出去了。」

且說那婆子早晨出門,來到西門慶門首探問,都說不知道。在對門牆腳下等夠多時,只見傅伙計來開鋪子。婆子走向前,道了萬福:「動問一聲,大官人在家麼?」傅伙計道:「你老人家尋他怎的?早是問著我,第二個也不知他。大官人昨日壽誕,在家請客,吃了一日酒,到晚拉眾朋友往院里去了,一夜通沒回家。你往那裡去尋他!」這婆子拜辭,出縣前來到東街口,正往勾欄那條巷去。只見西門慶騎著馬遠遠從東來,兩個小廝跟隨,此時宿酒未醒,醉眼摩娑,前合後仰。被婆子高聲叫道:「大官人,少吃些兒怎的!」向前一把手把馬嚼環扯住。西門慶醉中問道:「你是王乾娘,你來想是六姐尋我?」那婆子向他耳畔低言。道不數句,西門慶道:「小廝來家對我說來,我知道六姐惱我哩,我如今就去。」那西門慶一面跟著他,兩個一遞一句,整說了一路話。

比及到婦人門首,婆子先入去,報道:「大娘子恭喜,還虧老身,沒半個時辰,把大官人請將來了。」婦人聽見他來,就象天上掉下來的一般,連忙出房來迎接。西門慶搖著扇兒進來,帶酒半酣,與婦人唱喏。婦人還了萬福,說道:「大官人,貴人稀見面!怎的把奴丟了,一向不來傍個影兒?家中新娘子陪伴,如膠似漆,那裡想起奴家來!」西門慶道:「你休聽人胡說,那討什麼新娘子來!因小女出嫁,忙了幾日,不曾得閑工夫來看你。」婦人道:「你還哄我哩!你若不是憐新棄舊,另有別人,你指著旺跳身子說個誓,我方信你。」西門慶道:「我若負了你,生碗來大疔瘡,害三五年黃病,匾擔大蛆叮口袋。」婦人道:「負心的賊!匾擔大蛆叮口袋,管你甚事?」一手向他頭上把一頂新纓子瓦楞帽兒撮下來,望地上只一丟。慌的王婆地下拾起來,替他放在桌上,說道:「大娘子,只怪老身不去請大官人,來就是這般的。」婦人又向他頭上拔下一根簪兒,拿在手裡觀看,卻是一點油金簪兒,上面鈒著兩溜字兒:「金勒馬嘶芳草地,玉樓人醉杏花天。」卻是孟玉樓帶來的。婦人猜做那個唱的送他的,奪了放在袖子里,說道:「你還不變心哩!奴與你的簪兒那裡去了?」西門慶道:「你那根簪子,前日因酒醉跌下馬來,把帽子落了,頭髮散開,尋時就不見了。」婦人將手在向西門慶臉邊彈個響榧子,道:「哥哥兒,你醉的眼恁花了,哄三歲孩兒也不信!」王婆在旁插口道:「大娘子休怪!大官人,他離城四十里見蜜蜂兒刺屎,出門交獺象絆了一交,原來覷遠不覷近。」西門慶道:「緊自他麻犯人,你又自作耍。」婦人見他手中拿著一把紅骨細灑金、金釘鉸川扇兒,取過來迎亮處只一照,原來婦人久慣知風月中事,見扇上多是牙咬的碎眼兒,就疑是那個妙人與他的。不由分說,兩把折了。西門慶救時,已是扯的爛了,說道:「這扇子是我一個朋友卜志道送我的,一向藏著不曾用,今日才拿了三日,被你扯爛了。」

那婦人奚落了他一回,只見迎兒拿茶來,便叫迎兒放下茶托,與西門慶磕頭。王婆道:「你兩口子咭聒了這半日也夠了,休要誤了勾當。老身廚下收拾去也。」婦人一邊吩咐迎兒,將預先安排下與西門慶上壽的酒餚,整理停當,拿到房中,擺在桌上。婦人向箱中取出與西門慶上壽的物事,用盤盛著,擺在面前,與西門慶觀看。卻是一雙玄色段子鞋;一雙挑線香草邊闌、松竹梅花歲寒三友醬色段子護膝;一條紗綠潞綢、水光絹里兒紫線帶兒,裡面裝著排草玫瑰花兜肚;一根並頭蓮瓣簪兒。簪兒上鈒著五言四句詩一首,雲:「奴有並頭蓮,贈與君關髻。凡事同頭上,切勿輕相棄。」西門慶一見滿心歡喜,把婦人一手摟過,親了個嘴,說道:「怎知你有如此聰慧!」婦人教迎兒執壺斟一杯與西門慶,花枝招揚,插燭也似磕了四個頭。那西門慶連忙拖起來。兩個並肩而坐,交杯換盞飲酒。那王婆陪著吃了幾杯酒,吃的臉紅紅的,告辭回家去了。二人自在取樂玩耍。婦人陪伴西門慶飲酒多時,看看天色晚來,但見:
\begin{quote}
密雲迷晚岫,暗霧鎖長空。群星與皓月爭輝,綠水共青天同碧。僧投古寺,深林中嚷嚷鴉飛;客奔荒村,閭巷內汪汪犬吠。
\end{quote}

當下西門慶吩咐小廝回馬家去,就在婦人家歇了。到晚夕,二人儘力盤桓,淫慾無度。

常言道:樂極生悲。光陰迅速,單表武鬆自領知縣書禮馱擔,離了清河縣,竟到東京朱太尉處,下了書禮,交割了箱馱。等了幾日,討得回書,領一行人取路回山東而來。去時三四月天氣,回來卻淡暑新秋,路上雨水連綿,遲了日限。前後往回也有三個月光景。在路上行往坐臥,只覺得神思不安,身心恍惚,不免先差了一個土兵,預報與知縣相公。又私自寄一封家書與他哥哥武大,說他只在八月內準還。那土兵先下了知縣相公稟帖,然後逕來抓尋武大家。可可天假其便,王婆正在門首。那土兵見武大家門關著,才要叫門,婆子便問:「你是尋誰的?」土兵道:「我是武都頭差來下書與他哥哥。」婆子道:「武大郎不在家,都上墳去了。你有書信,交與我,等他回來,我遞與他,也是一般。」那土兵向前唱了一個喏,便向身邊取出家書來交與王婆,忙忙騎上頭口去了。

這王婆拿著那封書,從後門走過婦人家來。原來婦人和西門慶狂了半夜,約睡至飯時還不起來。王婆叫道:「大官人、娘子起來,和你們說話。如今武二差土兵寄書來與他哥哥,說他不久就到。我接下,打發他去了。你們不可遲滯,須要早作長便。」那西門慶不聽萬事皆休,聽了此言,正是:
\begin{quote}
分門八塊頂梁骨,傾下半桶冰雪來。
\end{quote}
慌忙與婦人都起來,穿上衣服,請王婆到房內坐下。取出書來與西門慶看。書中寫著,不過中秋回家。二人都慌了手腳,說道:「如此怎了?乾娘遮藏我每則個,恩有重報,不敢有忘。我如今二人情深似海,不能相舍。武二那廝回來,便要分散,如何是好?」婆子道:「大官人,有什麼難處之事!我前日已說過,幼嫁由親,後嫁由身。古來叔嫂不通門戶,如今武大已百日來到,大娘子請上幾個和尚,把這靈牌子燒了。趁武二未到家,大官人一頂轎子娶了家去。等武二那廝回來,我自有話說。他敢怎的?自此你二人自在一生,豈不是妙!」西門慶便道:「乾娘說的是。」當日西門慶和婦人用畢早飯,約定八月初六日,是武大百日,請僧燒靈。初八日晚,娶婦人家去。三人計議已定。不一時,玳安拿馬來接回家,不在話下。

光陰似箭,日月如梭,又早到了八月初六日。西門慶拿了數兩碎銀錢,來婦人家,教王婆報恩寺請了六個僧,在家做水陸,超度武大,晚夕除靈。道人頭五更就挑了經擔來,鋪陳道場,懸掛佛像。王婆伴廚子在竈上安排齋供。西門慶那日就在婦人家歇了。不一時,和尚來到,搖響靈杵,打動鼓鈸,諷誦經懺,宣揚法事,不必細說。

且說潘金蓮怎肯齋戒,陪伴西門慶睡到日頭半天,還不起來。和尚請齋主拈香僉字,證盟禮佛,婦人方纔起來梳洗,喬素打扮,來到佛前參拜。眾和尚見了武大這老婆,一個個都迷了佛性禪心,關不住心猿意馬,七顛八倒,酥成一塊。但見:
\begin{quote}
班首輕狂,念佛號不知顛倒;維摩昏亂,誦經言豈顧高低。燒香行者,推倒花瓶;秉燭頭陀,誤拿香盒。宣盟表白,大宋國錯稱做大唐國;懺罪闍黎,武大郎幾念武大娘。長老心忙,打鼓借拿徒弟手;沙彌情盪,罄槌敲破老僧頭。從前苦行一時休,萬個金剛降不住。
\end{quote}

婦人在佛前燒了香,僉了字,拜禮佛畢,回房去依舊陪伴西門慶。擺上酒席葷腥,自去取樂。西門慶吩咐王婆:「有事你自答應便了,休教他來聒噪六姐。」婆子哈哈笑道:「你兩口兒只管受用,由著老娘和那禿廝纏。」

且說從和尚見了武大老婆喬模喬樣,多記在心裡。到午齋往寺中歇晌回來,婦人正和西門慶在房裡飲酒作歡。原來婦人臥房與佛堂止隔一道板壁。有一個僧人先到,走在婦人窗下水盆里洗手,忽聽見婦人在房裡顫聲柔氣,呻呻吟吟,哼哼唧唧,恰似有人交媾一般。遂推洗手,立住腳聽。只聽得婦人口裡喘聲呼叫:「達達,你只顧𢵞打到幾時?只怕和尚來聽見。饒了奴,快些丟了罷!」西門慶道:「你且休慌!我還要在蓋子上燒一下兒哩!」不想都被這禿廝聽了個不亦樂乎。落後眾和尚到齊了,吹打起法事來,一個傳一個,都知婦人有漢子在屋裡,不覺都手之舞之,足之蹈之。臨佛事完滿,晚夕送靈化財出去,婦人又早除了孝髻,登時把靈牌並佛燒了。那賊禿冷眼瞧見,帘子里一個漢子和婆娘影影綽綽並肩站著,想起白日里聽見那些勾當,只顧亂打鼓𢵞鈸不住。被風把長老的僧伽帽刮在地上,露出青旋旋光頭,不去拾,只顧𢵞鈸打鼓,笑成一塊。王婆便叫道:「師父,紙馬已燒過了,還只顧𢵞打怎的?」和尚答道:「還有紙爐蓋子上沒燒過。」西門慶聽見,一面令王婆快打發襯錢與他。長老道:「請齋主娘子謝謝。」婦人道:「乾娘說免了罷。」眾和尚道:「不如饒了罷。」一齊笑的去了。正是:
\begin{quote}
隔牆須有耳,窗外豈無人!
\end{quote}
有詩為證:
\begin{quote}
淫婦燒靈志不平,闍黎竊壁聽淫聲。
果然佛法能消罪,亡者聞之亦慘魂。
\end{quote}
