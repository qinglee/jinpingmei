
\chapter{玉簫跪受三章約 書童私掛一帆風}

詩曰:
\begin{quote}
玉殞珠沉思悄然,明中流淚暗相憐。
常圖蛺蝶花樓下,記效鴛鴦翠幕前。
只有夢魂能結雨,更無心緒學非煙。
朱顏皓齒歸黃土,脈脈空尋再世緣。
\end{quote}

話說眾人散了,已有雞唱時分,西門慶歇息去了。玳安拿了一大壺酒、幾碟下飯,在鋪子里還要和傅伙計、陳敬濟同吃。傅伙計老頭子熬到這咱,已是坐不住,搭下鋪就倒在炕上,向玳安道:「你自和平安吃罷,陳姐夫想也不來了。」玳安叫進平安來,兩個把那酒你一鐘我一盞都吃了。收過傢伙,平安便去門房裡睡了。玳安一面關上鋪子門,上炕和傅伙計兩個對廝腳兒睡下。傅伙計因閑話,向玳安說道:「你六娘沒了,這等棺槨念經發送,也夠他了。」玳安道:「他的福好,只是不長壽。俺爹饒使了這些錢,還使不著俺爹的哩。俺六娘嫁俺爹,瞞不過你老人家,他帶了多少帶頭來!別人不知道,我知道。銀子休說,只金珠玩好、玉帶、絛環、鬏髻、值錢的寶石,也不知有多少。為甚俺爹心裡疼?不是疼人,是疼錢。若說起六娘的性格兒,一家子都不如他,又謙讓又和氣,見了人,只是一面兒笑,自來也不曾喝俺每一喝,並沒失口罵俺每一句『奴才』。使俺每買東西,只拈塊兒。俺每但說:『娘,拿等子,你稱稱。』他便笑道:『拿去罷,稱什麼。你不圖落圖什麼來?只要替我買值著。』這一家子,那個不借他銀使?只有借出來,沒有個還進去的。還也罷,不還也罷。俺大娘和俺三娘使錢也好。只是五娘和二娘,慳吝的緊。他當家,俺每就遭瘟來。會勝買東西,也不與你個足數,綁著鬼,一錢銀子,只稱九分半,著緊只九分,俺每莫不賠出來!」傅伙計道:「就是你大娘還好些。」玳安道:「雖故俺大娘好,毛司火性兒,一回家好,娘兒每親親噠噠說話兒,你只休惱著他,不論誰,他也罵你幾句兒。總不如六娘,萬人無怨,又常在爹跟前替俺每說方便兒。隨問天來大事,俺每央他央兒對爹說,無有個不依。只是五娘,行動就說:『你看我對爹說不說!』把這打只提在口裡。如今春梅姐,又是個合氣星。——天生的都在他一屋裡。」傅伙計道:「你五娘來這裡也好幾年了。」玳安道:「你老人家是知道的,想的起他那咱來的光景哩。他一個親娘也不認的,來一遭,要便搶的哭了家去。如今六娘死了,這前邊又是他的世界,明日那個管打掃花園,乾凈不乾凈,還吃他罵的狗血噴了頭哩!」兩個說了一回,那傅伙計在枕上齁齁就睡著了。玳安亦有酒了,合上眼,不知天高地下,直至紅日三竿,都還未起來。

原來西門慶每常在前邊靈前睡,早晨玉簫出來收疊床鋪,西門慶便往後邊梳頭去。書童蓬著頭,要便和他兩個在前邊打牙犯嘴,互相嘲逗,半日才進後邊去。不想這日西門慶歸上房歇去,玉簫趕人沒起來,暗暗走出來,與書童約了,走在花園書房裡乾營生去了。不料潘金蓮起的早,驀地走到廳上,只見靈前燈兒也沒了,大棚里丟的桌椅橫三豎四,沒一個人兒,只有畫童兒在那裡掃地。金蓮道:「賊囚根子,乾凈只你在這裡,都往那裡去了?」畫童道:「他每都還沒起來哩。」金蓮道:「你且丟下笤帚,到前邊對你姐夫說,有白絹拿一匹來,你潘姥姥還少一條孝裙子,再拿一副頭須系腰來與他。他今日家去。」畫童道:「怕不俺姐夫還睡哩,等我問他去。」良久回來道:「姐夫說不是他的首尾,書童哥與崔本哥管孝帳。娘問書童哥要就是了。」金蓮道:「知道那奴才往那去了,你去尋他來。」畫童向廂房裡瞧了瞧,說道:「才在這裡來,敢往花園書房裡梳頭去了。」金蓮說道:「你自掃地,等我自家問這囚根子要去。」因走到花園書房內,忽然聽見裡面有人笑聲。推開門,只見書童和玉簫在床上正幹得好哩。便罵道:「好囚根子,你兩個幹得好事!」唬得兩個做手腳不迭,齊跪在地下哀告。金蓮道:「賊囚根子,你且拿一匹孝絹、一匹布來,打發你潘姥姥家去著。」書童連忙拿來遞上。金蓮逕歸房來。

那玉簫跟到房中,打旋磨兒跪在地下央及:「五娘,千萬休對爹說。」金蓮便問:「賊狗肉,你和我實說,從前已往,偷了幾遭?一字兒休瞞我,便罷。」那玉簫便把和他偷的緣由說了一遍。金蓮道:「既要我饒你,你要依我三件事。」玉簫道:「娘饒了我,隨問幾件事我也依娘。」金蓮道:「第一件,你娘房裡,但凡大小事兒,就來告我說。你不說,我打聽出來,定不饒你。第二件,我但問你要甚麼,你就捎出來與我。第三件,你娘向來沒有身孕,如今他怎生便有了?」玉簫道:「不瞞五娘說,俺娘如此這般,吃了薛姑子的衣胞符藥,便有了。」潘金蓮一一聽記在心,才不對西門慶說了。

書童見潘金蓮冷笑領進玉簫去了,知此事有幾分不諧。向書房廚櫃內收拾了許多手帕汗巾、挑牙簪紐,並收的人情,他自己也攢有十來兩銀子,又到前邊柜上誆了傅伙計二十兩,只說要買孝絹,逕出城外,雇了長行頭口,到碼頭上,搭在鄉裡船上,往蘇州原籍家去了。正是:
\begin{quote}
撞碎玉籠飛彩鳳,頓開金鎖走蛟龍。
\end{quote}

那日,李桂姐、吳銀兒、鄭愛月都要家去了。薛內相、劉內相早晨差人抬三牲桌面來祭奠燒紙。又每人送了一兩銀子伴宿分資,叫了兩個唱道情的來,白日里要和西門慶坐坐。緊等著要打發孝絹,尋書童兒要鑰匙,一地裡尋不著。傅伙計道:「他早晨問我柜上要了二十兩銀子買孝絹去了,口稱爹吩咐他孝絹不夠,敢是向門外買去了?」西門慶道:「我並沒吩咐他,如何問你要銀子?」一面使人往門外絹鋪找尋,那裡得來!月娘向西門慶說:「我猜這奴才有些蹺蹊,不知弄下甚麼硶兒,拐了幾兩銀子走了。你那書房裡還大瞧瞧,只怕還拿甚麼去了。」西門慶走到兩個書房裡都瞧了,只見庫房裡鑰匙掛在牆上,大櫥櫃里不見了許多汗巾手帕,並書禮銀子、挑牙紐扣之類,西門慶心中大怒,叫將該地方管役來,吩咐:「各處三街兩巷與我訪緝。」那裡得來!正是:
\begin{quote}
不獨懷家歸興急,五湖煙水正茫茫。
\end{quote}

那日,薛內相從晌午就坐轎來了。西門慶請下吳大舅、應伯爵、溫秀才相陪。先到靈前上香,打了個問訊,然後與西門慶敘禮,說道:「可傷,可傷!如夫人是甚病兒歿了?」西門慶道:「不幸患崩瀉之疾歿了,多謝老公公費心。」薛內相道:「沒多兒,將就表意罷了。」因看見掛的影,說道:「好位標緻娘子!正好青春享福,只是去世太早些。」溫秀才在旁道:「物之不齊,物之情也。窮通壽夭,自有個定數,雖聖人亦不能強。」薛內相扭回頭來,見溫秀才穿著衣巾,因說道:「此位老先兒是那學里的?」溫秀才躬身道:「學生不才,備名府庠。」薛內相道:「我瞧瞧娘子的棺木兒。」西門慶即令左右把兩邊帳子撩起,薛內相進去觀看了一遍,極口稱贊道:「好副板兒!請問多少價買的?」西門慶道:「也是舍親的一副板,學生回了他的來了。」應伯爵道:「請老公公試估估,那裡地道,甚麼名色?」薛內相仔細看了說:「此板不是建昌,就是副鎮遠。」伯爵道:「就是鎮遠,也值不多。」薛內相道:「最高者,必定是楊宣榆。」伯爵道:「楊宣榆單薄短小,怎麼看得過!此板還在楊宣榆之上,名喚做桃花洞,在於湖廣武陵川中。昔日唐漁父入此洞中,曾見秦時毛女在此避兵,是個人跡罕到之處。此板七尺多長,四寸厚,二尺五寬。還看一半親家分上,還要了三百七十兩銀子哩。公公,你不曾看見,解開噴鼻香的,裡外俱有花色。」薛內相道:「是娘子這等大福,才享用了這板。俺每內官家,到明日死了,還沒有這等發送哩。」吳大舅道:「老公公好說,與朝廷有分的人,享大爵祿,俺們外官焉能趕的上。老公公日近清光,代萬歲傳宣金口。見今童老爺加封王爵,子孫皆服蟒腰玉,何所不至哉!」薛內相便道:「此位會說話的兄,請問上姓?」西門慶道:「此是妻兄吳大哥,見居本衛千戶之職。」薛內相道:「就是此位娘子令兄麼?」西門慶道:「不是。乃賤荊之兄。」薛內相復於吳大舅聲諾說道:「吳大人,失瞻!」

看了一回,西門慶讓至捲棚內,正面安放一把交椅,薛內相坐下,打茶的拿上茶來吃了。薛內相道:「劉公公怎的這咱還不到?叫我答應的迎迎去。」青衣人跪下稟道:「小的邀劉公公去來,劉公公轎已伺候下了,便來也。」薛內相又問道:「那兩個唱道情的來了不曾?」西門慶道:「早上就來了。——叫上來!」不一時,走來面前磕頭。薛內相道:「你每吃了飯不曾?」那人道:「小的每吃了飯了。」薛內相道:「既吃了飯,你每今日用心答應,我重賞你。」西門慶道:「老公公,學生這裡還預備著一起戲子,唱與老公公聽。」薛內相問:「是那裡戲子?」西門慶道:「是一班海鹽戲子。」薛內相道:「那蠻聲哈剌,誰曉的他唱的是甚麼!那酸子每在寒窗之下,三年受苦,九載遨遊,背著琴劍書箱來京應舉,得了個官,又無妻小在身邊,便希罕他這樣人。你我一個光身漢、老內相,要他做甚麼?」溫秀才在旁邊笑說道:「老公公說話,太不近情了。居之齊則齊聲,居之楚則楚聲。老公公處於高堂廣廈,豈無一動其心哉?」這薛內相便拍手笑將起來道:「我就忘了溫先兒在這裡。你每外官,原來只護外官。」溫秀才道:「雖是士大夫,也只是秀才做的。老公公砍一枝損百林,兔死狐悲,物傷其類。」薛內相道:「不然。一方之地,有賢有愚。」

正說著,忽左右來報:「劉公公下轎了。」吳大舅等出去迎接進來,向靈前作了揖。敘禮已畢,薛內相道:「劉公公,你怎的這咱才來?」劉內相道:「北邊徐同家來拜望,陪他坐了一回,打發去了。」一面分席坐下,左右遞茶上去。因問答應的:「祭奠桌面兒都擺上了不曾?」下邊人說:「都排停當了。」劉內相道:「咱每去燒了紙罷。」西門慶道:「老公公不消多禮,頭裡已是見過禮了。」劉內相道:「此來為何?還當親祭祭。」當下,左右捧過香來,兩個內相上了香,遞了三鐘酒,拜下去。西門慶道:「老公公請起。」於是拜了兩拜起來,西門慶還了禮,復至捲棚內坐下。然後收拾安席,遞酒上坐。兩位內相分左右坐了,吳大舅、溫秀才、應伯爵從次,西門慶下邊相陪。子弟鼓板響動,遞了關目揭帖。兩位內相看了一回,揀了一段《劉智遠白兔記》。唱了還未幾折,心下不耐煩,一面叫上兩個唱道情的去,打起漁鼓,並肩朝上,高聲唱了一套「韓文公雪擁藍關」故事下去。

薛內相便與劉內相兩個說說話兒,道:「劉哥,你不知道,昨日這八月初十日,下大雨如註,雷電把內里凝神殿上鴟尾裘碎了,唬死了許多宮人。朝廷大懼,命各官修省,逐日在上清宮宣《精靈疏》建醮。禁屠十日,法司停刑,百官不許奏事。昨日大金遣使臣進表,要割內地三鎮,依著蔡京那老賊,就要許他。掣童掌事的兵馬,交都御史譚積、黃安十大使節制三邊兵馬,又不肯,還交多官計議。昨日立冬,萬歲出來祭太廟,太常寺一員博士,名喚方軫,早晨打掃,看見太廟磚縫出血,殿東北上地陷了一角,寫表奏知萬歲。科道官上本,極言童掌事大了,宦官不可封王。如今馬上差官,拿金牌去取童掌事回京。」劉內相道:「你我如今出來在外做土官,那朝事也不乾咱每。俗語道,咱過了一日是一日。便塌了天,還有四個大漢。到明天,大宋江山管情被這些酸子弄壞了。王十九,咱每隻吃酒!」因叫唱道情的上來,吩咐:「你唱個『李白好貪杯』的故事。」那人立在席前,打動漁鼓,又唱了一回。

直吃至日暮時分,吩咐下人,看轎起身。西門慶款留不住,送出大門,喝道而去。回來,吩咐點起燭來,把桌席休動,留下吳大舅、應伯爵、溫秀才坐的,又使小廝請傅伙計、甘伙計、韓道國、賁第傳、崔本和陳敬濟復坐。叫上子弟來吩咐:「還找著昨日《玉環記》上來。」因向伯爵道:「內相家不曉的南戲滋味。早知他不聽,我今日不留他。」伯爵道:「哥,到辜負你的意思。內臣斜局的營生,他只喜《藍關記》、搗喇小子山歌野調,那裡曉的大關目悲歡離合!」於是下邊打動鼓板,將昨日《玉環記》做不完的折數,一一緊做慢唱,都搬演出來。西門慶令小廝席上頻斟美酒。伯爵與西門慶同桌而坐,便問:「他姐兒三個還沒家去,怎的不叫出來遞杯酒兒?」西門慶道:「你還想那一夢兒,他每去的不耐煩了!」伯爵道:「他每在這裡住了有兩三日?」西門慶道:「吳銀兒住的久了。」當日,眾人坐到三更時分,搬戲已完,方起身各散。西門慶邀下吳大舅,明日早些來陪上祭官員。與了戲子四兩銀子,打發出門。

到次日,周守備、荊都監、張團練、夏提刑,合衛許多官員,都合了分資,辦了一副豬羊吃桌祭奠,有禮生讀祝。西門慶預備酒席,李銘等三個小優兒伺候答應。到晌午,只聽鼓響,祭禮到了。吳大舅、應伯爵、溫秀才在門首迎接,只見後擁前呼,眾官員下馬,在前廳換衣服。良久,把祭品擺下,眾官齊到靈前,西門慶與陳敬濟還禮。禮生喝禮,三獻畢,跪在旁邊讀祝,祭畢。西門慶下來謝禮已畢,吳大舅等讓眾官至捲棚內,寬去素服,待畢茶,就安席上坐,觥籌交錯,殷勤勸酒。李銘等三個小優兒,銀箏檀板,朝上彈唱。眾官歡飲,直到日暮方散。西門慶還要留吳大舅眾人坐,吳大舅道:「各人連日打攪,姐夫也辛苦了,各自歇息去罷。」當時告辭回家。正是:
\begin{quote}
天上碧桃和露種,日邊紅杏倚雲栽。
家中巨富人趨附,手內多時莫論財。
\end{quote}
